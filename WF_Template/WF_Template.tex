% XeLaTeX

\documentclass[landscape,a4paper]{article}
\usepackage{ctex}
\usepackage{xypic}
\usepackage{amsfonts,amssymb}
\usepackage{multirow}
\usepackage{geometry}
\usepackage{graphicx}
\usepackage{listings}
\usepackage{courier}
\usepackage{etoolbox}
\usepackage{amsmath}
\usepackage{pifont}
% \usepackage{fancyhdr}
\usepackage{titlesec}


%\pagestyle{fancy}
%\fancyhead[L]{Zhongshan (Sun Yat-sen) University}
%\fancyhead[R]{\thepage}


\newpagestyle{main}{            
    \sethead{Zhongshan (Sun Yat-sen) University}{}{\thepage}
    \headrule
}
\pagestyle{main}


% \linespread{1.2}
\geometry{left=1.5cm,right=1cm,top=1.5cm,bottom=0.5cm}

\makeatletter
\patchcmd{\FV@SetupFont}
  {\FV@BaseLineStretch}
  {\fontencoding{T1}\FV@BaseLineStretch}
  {}{}
\makeatother

\lstset{basicstyle=\footnotesize\fontencoding{T1}\ttfamily,breaklines=true}
\lstset{numbers=left,frame=single,tabsize=4,language=C++}
%\lstset{extendedchars=false}
\begin{document}

\small
\titleformat{\section}{\normalsize\bfseries}{\thesection}{1em} {}
\titleformat{\subsection}{\normalsize\bfseries}{\thesubsection}{1em} {}
\twocolumn
\setlength{\columnsep}{1.5cm}

\title{ICPC World Finals 2019 Templates}
\author {SYSU\_Balloon}
\maketitle
\thispagestyle{main}
\tableofcontents

\setlength{\parskip}{-0.5em}
\setlength{\lineskip}{-0.5em}

\section{数学}

\subsection{结论和定理}

\paragraph{五边形定理} 五边形数 $n(3n \pm 1)/2$ 。$(1-x)(1-x^2)(1-x^3)\cdots = \Sigma{(-1)^kx^{n(3n \pm 1)/2}}$ ,即 $f(n) = f(n - 1) + f(n - 2) - f(n - 5) - f(n - 7) + f(n - 12) + f(n - 15) - \cdots$ 。

\paragraph{斐波那契数性质} 
\ding{172} $f[n] = f[n - 1] + f[n - 2]$ ; 
\ding{173} $f[n + m + 1] = f[n]f[m] + f[n+1]f[m+1]$; 
\ding{174} $\gcd{(f[n], f[n+1])}=1$; 
\ding{175} $\gcd{(f[n], f[n+2])}=1$; 
\ding{176} $\gcd{(f[n], f[m])}=f[\gcd{(n, m)}]$ ; 
\ding{177} $f[n+1]^2-f[n]f[n+2]=(-1)^n$; 
\ding{178} $\Sigma_{i=1}^{n} f[i]^2=f[n]f[n+1]$; 
\ding{179} $\Sigma_{i=0}^{n} f[i]=f[n+2]-1$; 
\ding{180} $\Sigma_{i=1}^{n} f[2i-1]=f[2n]$; 
\ding{181} $\Sigma_{i=1}^{n} f[2i]=f[2n+1]-1$; 
\ding{182} $\Sigma_{i=0}^{n}(-1)^if[i]=(-1)^n(f[n+1]-f[n])+1$; 
\ding{183} $f[2n-1]=f[n]^2-f[n-2]^2$; 
\ding{184} $f[2n+1]=f[n]^2+f[n+1]^2$; 
\ding{185} $3f[n] = f[n+2] + f[n-2]$; 
\ding{186} $f[n]=\Sigma_{i=0}^{m} \binom{n-1-i}{i} (m\leq n-1-m)$; 
\ding{187} $\Sigma_{i=1}^{n}if[i]=nf[n+2]-f[n+3]+2$ .

\paragraph{卡特兰数性质} 
\ding{172} 凸多边形三角剖分数;
\ding{173} 简单有序根树的计数;
\ding{174} $(0, 0)$ 走到 $(n, n)$ 经过的点 $(a, b)$ 满足 $a \leq b$ 的方案数;
\ding{175} $h_1 = 1, h_n = \frac{h_{n-1}(4n-2)}{n+1} = \frac{C(2n, n)}{n+1} = C(2n, n) - C(2n, n-1)$ ;
\ding{176} 在一个格点阵列中,从 $(0, 0)$ 点走到 $(n, m)$ 点且不经过/穿过对角线 $x = y$ 的方案数 : $\binom{n+m-1}{m} - \binom{n+m-1}{m-1} (x > y)$ ;  $\binom{n+m}{m} - \binom{n+m}{m-1} (x \geq y)$ 

\paragraph{第一类斯特林数性质}
\ding{172} 有正有负,其绝对值是 $n$ 个元素的项目分作 $k$ 个非空循环排列的数量 $s[n][k]$ ;
\ding{173} $s[n][0] = 0 (n \geq 1), s[n][n] = 1 (n \geq 0)$ ;
\ding{174} $s[n][k] = (n-1)s[n-1][k] + s[n - 1][k - 1] (1 \leq k \leq n - 1)$ ;
\ding{175} $|s[n][1]| = (n-1)!$ ;
\ding{176} $s[n][k] = (-1)^{n+k}|s[n][k]|$ ;
\ding{177} $s[n][n-1] = -\binom{n}{2}$;
\ding{178} $x(x-1)(x-2) \cdots (x-n+1) = \Sigma{s[n][k]x^k}$.

\paragraph{第二类斯特林数性质}
\ding{172} 将 $n$ 个物体划分为 $k$ 个非空的不可辨别(可理解为盒子没有编号)集合的方法数;
\ding{173} $s[n][0] = 0 (n \geq 1), s[n][n] = 1 (n \leq 0)$ ;
\ding{174} $s[n][k] = ks[n-1][k] + s[n-1][k-1]$;
\ding{175} $s[n][n - 1] = \binom{n}{2}$
\ding{176} $s[n][2] = 2^{n-1}-1$
\ding{177} $s[n][k] = \frac{1}{k!}\Sigma_{i=0}^{k} (-1)^i\binom{k}{i} (k-i)^n$ .

\paragraph{Bell 数性质}
\ding{172} $n$ 个元素的集合划分数目;
\ding{173} $B[n] = \Sigma_{k=1}^{n} s[n][k]$;
\ding{174} $B[n+1] = \Sigma_{k=0}^{n} \binom{n}{k}B[k]$;
\ding{175} $B[n+p]=B[n]+B[n+1] (\text{mod }p)$.

\paragraph{多项式性质}
\ding{172} $f(x)$ 不存在重根 $\Leftrightarrow \gcd{(f(x), f'(x))}$ 的次数小于1次;
\ding{173} 多项式 $\gcd$ 可以用来判断两多项式是否有公共根.

\paragraph{多项式取模}
$f(x) = 0 (\text{mod }m_0), m_0 = \Pi_{i=1}^{k}m_i$ . 用 $T_i$ 表示 $f(x) = 0 (\text{mod }m_i)$ 的解数,则 $T_0 = \Pi_{i=1}^{k}T_i$ .

\paragraph{数论}
\ding{172} $a^n \text{mod }b = a^{n \text{mod }\phi{(b)}+\phi{(b)}} \text{mod }b (n \geq \phi{(b)})$;
\ding{173} lucas 定理 $\binom{n}{m} = \binom{n\%p}{m\%p}\binom{n/p}{m/p} (\text{mod }p)$;
\ding{174} lucas 函数 $f(n, m) = f(n\%p,m\%p)f(n/p, m/p) (\text{mod }p)$,可以猜测满足.

\paragraph{原根}
\ding{172} 2, 4, $p^k$, $2p^k$ 存在原根,存在原根则原根数量为 $\phi(\phi(n))$ ;
\ding{173} 验证原根 $x = \phi{(n)}, x = p_1^{a_1}p_2^{a_2}\cdots p_k^{a_k}$,原根满足 $t^{x/p_i} \neq 1 (\text{mod }n)$

\paragraph{$x^2+y^2=n$ 的整数解}
解的个数为 $4\Sigma_{d|n}H(d)$. $H(d)=d\%2 ? (-1)^{(d-1)/2} : 0$ .

\paragraph{平方和定理}
\ding{172} 奇质数能表示为两个平方数之和的充分必要条件是该质数被4除余1.
\ding{173} 正整数能表示为两个平方数之和的充要条件是在它的标准分解式中,形如素因子的指数是偶数.
\ding{174} 如果两个整数都能表示为两个平方数之和,则它们的积也能表示为两个平方数之和: $(a^2+b^2)(c^2+d^2)=(ac-bd)^2+(ad+bc)^2=(ac+bd)^2+(ad-bc)^2$.
\ding{175} 每个正整数都可以表示成4个整数的平方数之和.

\paragraph{杨氏矩阵}
\ding{172} 如果格子 $(i, j)$ 没有元素,则它右边和上面的相邻格子也一定没有元素;
\ding{173} 如果格子 $(i, j)$ 有元素 $a[i][j]$ ,则它右边和上面的相邻格子要么没有元素,要么有元素且比 $a[i][j]$ 大.
\ding{174} $1 \cdots n$ 所组成杨氏矩阵的个数可以通过下面的递推式得到: $f[1] = 1, f[2] = 2, f[n] = f[n - 1] + (n - 1)f[n - 2]$.
\ding{175} 钩子公式:对于给定形状,不同的杨氏矩阵的个数为: $n!$ 除以每个格子的钩子长度加 $1$ 的积,其中钩子长度定义为该格子右边的格子数和它上面的格子数之和.

\paragraph{蔡勒公式}
$w = (\lfloor \frac{c}{4} \rfloor - 2c + y + \lfloor \frac{y}{4} \rfloor + \lfloor \frac{13(m+1)}{5} \rfloor + d - 1) \text{ mod } 7$ 。
\ding{172} $w$ : 0星期日,1星期一,$\cdots$,6星期六;
\ding{173} $c$ : 年份前两位数;
\ding{174} $y$ : 年份后两位数;
\ding{175} $m$ : 月( $3 \leq m \leq 14$ ,即在蔡勒公式中,1、2月要看作上一年的13、14月来计算 );
\ding{176} $d$ : 日。

\paragraph{皮克定理}
给定顶点坐标均是整点(或正方形格点)的简单多边形(凸多边形),皮克定理说明了其面积 $S$ 和内部格点数目 $n$ 、边上格点数目 $s$ 的关系: $S = n + \frac{s}{2} + 1$ 。

\subsection{Miller Rabin}
\begin{lstlisting}
// 2,7,61 : < 4759123141
// 2,3,5,7,11,13,17 : < 341550071728320
// 2,3,7,61,24251 : < 10^16 only 46856248255981

//BZOJ-3667
#include<cstdio>
#include<cstdlib>
typedef long long ll;
ll _,n,x,ans,st;
ll gcd(ll x,ll y){return y==0?x:gcd(y,x%y);}
#define abs(x) (x>0?x:-(x))
#define cmax(a,b) (a<b?a=b:1)
ll mul(ll a,ll b,ll p){
    ll tmp=(a*b-(ll)((long double)a/p*b+1e-7)*p);
    return tmp<0?tmp+p:tmp;
}
ll power(ll t,ll k,ll p){
    ll f=1;
    for(;k;k>>=1,t=mul(t,t,p))if(k&1)f=mul(f,t,p);
    return f;
}
bool check(ll a,int k,ll p,ll q){
    ll t=power(a,q,p);
    if(t==1||t==p-1)return 1;
    for(;k--;){
        t=mul(t,t,p);
        if(t==p-1)return 1;
    }
    return 0;
}
bool mr(ll p){
    if(p<=1)return 0;
    if(p==2)return 1;
    if(~p&1)return 0;
    ll q=p-1;int i,k=0;
    while(~q&1)q>>=1,k++;
    for(i=0;i<5;i++)
    if(!check(rand()%(p-1)+1,k,p,q))return 0;
    return 1;
}
ll rho(ll n,ll c){
    ll x=rand()%n,y=x,p=1;
    while(p==1)
        x=(mul(x,x,n)+c)%n,
        y=(mul(y,y,n)+c)%n,
        y=(mul(y,y,n)+c)%n,
        p=gcd(n,abs(x-y));
    return p;
}
void solve(ll n){
    if(n==1)return;
    if(mr(n)){cmax(ans,n);return;}
    if(~n&1)cmax(ans,2),solve(n>>1);
    else{
        ll t=n;
        while(t==n)t=rho(n,rand()%(n-1)+1);
        solve(t),solve(n/t);
    }
}
int main(){
    for(srand(1626),scanf("%lld",&_);_--;){
        scanf("%lld",&x),ans=0;solve(x);
        if(ans==x)puts("Prime");
        else printf("%lld\n",ans);
    }
}
\end{lstlisting}

\subsection{同余方程}
\begin{lstlisting}[language=C++]
void gcd(LL a, LL b, LL &d, LL &x, LL &y){
	if (!b){ d=a; x=1; y=0; return; }
	gcd(b,a%b,d,y,x); y-=x*(a/b);
}
IL void sim(LL &a, LL n) { a%=n; if (a<0) a+=n; }
IL LL solve(LL a, LL b, LL n){  // a*x==b (mod n)
	sim(a,n); sim(b,n);  // optional
	static LL d,x,y;
	gcd(a,n,d,x,y);
	if (b%d) return -1;
	b/=d; n/=d;
	if (x<0) x+=n;
	return b*x%n;
}
// x==a1 (mod n1); x==a2 (mod n2);
// passing gcd in solve can reduce time 
void merge(LL a1, LL n1, LL a2, LL n2, LL &x, LL &n){
	n=lcm(n1,n2);
	LL k=solve(n1,a2-a1,n2);
	if (k==-1) { x=-1; return; }
	sim(x=n1*k+a1,n);
}
// getinv , gcd(a,n) must be 1
IL LL getinv(LL a, LL n){
	static LL d,x,y;
	gcd(a,n,d,x,y);
	// if (d!=1) return -1;
	return x<0?x+n:x;
}
\end{lstlisting}

\subsection{线性筛法}
\begin{lstlisting}
const int N=100050;
int b[N],a[N],cnt,mx[N],phi[N],mu[N];

void getprime(int n=100000){
	memset(b+2,1,sizeof(b[0])*(n-1));
	mu[1]=1;
	ft(i,2,n){
		if (b[i]){
			a[mx[i]=++cnt]=i;
			phi[i]=i-1; mu[i]=-1;
		}
		ft(j,1,mx[i]){
			int k=i*a[j];
			if (k>n) break;
			b[k]=0; mx[k]=j;
			phi[k]= phi[i]*(a[j]-(j!=mx[i]));
			mu[k]= j==mx[i] ? 0 : -mu[i];
		}
	}
}
\end{lstlisting}

\subsection{离散对数}
\begin{lstlisting}
// BSGS , a^x==b (mod n) , n is a prime
LL bsgs(LL a, LL b, LL n){
	int m=sqrt(n+0.5);
	LL p=power(a,m,n);
	LL v=getinv(p,n);
	static hash_map x;
	x.clear();
	LL e=1; x[e]=0;
	ft(i,1,m){
		e=e*a%n;
		if (!x.count(e)) x[e]=i;
	}
	for(LL i=0;i<n;i+=m){
		if (x.count(b)) return i+x[b];
		b=b*v%n;
	}
	return -1;
}

//BSGS
//y^x==z (mod p) ->x=?
scanf("%d%d%d",&y,&z,&p),y%=p,z%=p;j=z;
if(y==0){puts("Cannot find x");continue;}
for(k=s=1;k*k<=p;k++);
std::map<int,int>hash;flag=0;
for(int i=0;i<k;i++,s=1LL*s*y%p,j=1LL*j*y%p)hash[j]=i;
for(int i=1,j=s;i<=k&&!flag;i++,j=1LL*j*s%p)
if(hash.count(j))ans=i*k-hash[j],flag=1;
if(flag==0)puts("Cannot find x");
else printf("%d\n",ans);

//exBSGS
int bsgs(int a,ll b,int p){
    if(a%=p,b%=p,b==1)return 0;
    ll t=1;int f,g,delta=0,m=sqrt(p)+1,i;
    for(g=gcd(a,p);g!=1;g=gcd(a,p)){
        if(b%g)return -1;
        b/=g,p/=g,t=t*(a/g)%p,delta++;
        if(b==t)return delta;
    }
    std::map<int,int>hash;
    for(i=0;i<m;i++,b=b*a%p)hash[b]=i;
    for(i=1,f=power(a,m);i<=m+1;i++)
    if(t=t*f%p,hash.count(t))return i*m-hash[t]+delta;
    return -1;
}
\end{lstlisting}

\subsection{Lucas}
\begin{lstlisting}
void init_Lucas(){
	fac[0]=1; ft(i,1,P-1) fac[i]=fac[i-1]*i%P;
	inv[1]=1; ft(i,2,P-1) inv[i]=(P-P/i)*inv[P%i]%P;
	inv[0]=1; ft(i,1,P-1) inv[i]=inv[i-1]*inv[i]%P;
}
IL LL C(int n, int m){
	LL ans=1;
	while (n||m){
		int a=n%P, b=m%P;
		if (a<b) return 0;
		n/=P; m/=P;
		ans= ans *fac[a]%P *inv[b]%P *inv[a-b]%P;
	}
	return ans;
}
\end{lstlisting}

\subsection{高斯消元法实数方程}
\begin{lstlisting}
void Gauss(int n,int m) {
	int i,j,k,t;
	double mul;
	for (i=j=1;i<=n&&j<=m;i++,j++) {
		for (k=i+1;k<=n;k++)
			if (abs(mat[k][j])>abs(mat[i][j]))
				for (t=1;t<=m+1;t++) swap(mat[i][t],mat[k][t]);
		if (abs(mat[i][j])<eps) { i--; continue; }
		for (k=i+1;k<=n;k++) {
			mul=mat[k][j]/mat[i][j];
			for (t=1;t<=m+1;t++) mat[k][t]-=mat[i][t]*mul;
		}
	}
	for (i=n;i>=1;i--) { //solved表示那个变量是否确定
		for (j=1;j<=m;j++) if (abs(mat[i][j])>eps) break;
		if  (j>m) continue; solved[j]=true; ans[j]=mat[i][m+1];
		for (k=j+1;k<=m;k++)
			if (abs(mat[i][k])>eps&&!solved[k]) solved[j]=false;
		for (k=j+1;k<=m;k++) ans[j]-=ans[k]*mat[i][k];
		ans[j]/=mat[i][j];
	}
}
\end{lstlisting}

\subsection{高斯消元解异或方程}
\begin{lstlisting}
int n,m;
bitset<N> a[N];
bool solve(){
	int i=1, j=1;	
	while (i<=n && j<=m){
		int k=i;
		while (k<=n && !a[k][j]) k++;
		if (k>n) { j++; continue; }
		if (j==m) return false;  // no solution
		if (k!=i) swap(a[i],a[k]);
		ft(t,1,n) if (t!=i && a[t][j]) a[t]^=a[i];
		i++; j++;
	}
	return true; // have solution (but may have 0==0)
}
\end{lstlisting}

\subsection{高斯消元法模方程}
\begin{lstlisting}
void Gauss(LL n,LL m) {
	LL i,j,k,t,lcm,muli,mulk;
	for (i=j=1;i<=n&&j<=m;i++,j++) {
		for (k=i;k<=n;k++) if (mat[k][j]) {
			for (t=1;t<=m+1;t++) swap(mat[k][t],mat[i][t]);
			break;
		}
		if (mat[i][j]==0) { i--; continue; }
		for (k=i+1;k<=n;k++) if (mat[k][j]) {
			lcm=mat[k][j]*mat[i][j]/__gcd(mat[k][j],mat[i][j]);
			muli=lcm/mat[i][j]; mulk=lcm/mat[k][j];
			for (t=1;t<=m+1;t++) {
				mat[k][t]=mat[k][t]*mulk-mat[i][t]*muli;
				mat[k][t]=(mat[k][t]%mod+mod)%mod;
			}
		}
	}
	for (i=n;i>=1;i--) {
		for (j=1;j<=m;j++) if (mat[i][j]) break;
		if  (j>m) continue; ans[j]=mat[i][m+1];
		for (k=j+1;k<=m;k++) ans[j]-=ans[k]*mat[i][k];
		ans[j]=(ans[j]*power(mat[i][j],mod-2)%mod+mod)%mod;
	}
}
\end{lstlisting}

\subsection{瀚之的莫比乌斯}
\begin{lstlisting}
void getprime(int n){
	miu[1]=pre[1]=b[1]=1;
	for(int i=2;i<=n;i++){
		if(!b[i]) p[mx[i]=++cnt]=i, miu[i]=-1;
		for(int j=1;j<=mx[i];j++){
			int k=i*p[j]; if(k>n) break;
			b[k]=1; mx[k]=j;
			if(j==mx[i]) miu[k]=0;
				else miu[k]=miu[i]*miu[p[j]];
		}
		pre[i]=pre[i-1]+i*miu[i];
	}
}
ll f(int n,int m){return 1ll*n*m*(n+m+2)/2;} 
ll calc(int n,int m){ // sigma{ i+j | i<=n,j<=m,gcd(i,j)==1 }
	if(n>m) swap(n,m); ll ans=0;
	for(int i=1,j=0,k=0;i<=n;i=min(j,k)+1)
		ans+=(pre[min(j=n/(n/i),k=m/(m/i))]-pre[i-1])*f(n/i,m/i);
	return ans;
}
\end{lstlisting}

\subsection{FFT|NTT}

\begin{lstlisting}
typedef complex<double> comp;
comp A[N], B[N];
int rev[N], m, len;
inline void init(int n) {
	for (m = 1, len = 0; m < n + n; m <<= 1 , len ++);
	for (int i = 0; i < m; ++i) rev[i]=(rev[i>>1]>>1) | ((i&1)<<(len-1));
	for (int i = 0; i < m; ++i) A[i] = B[i] = comp(0, 0);
}
inline void dft(comp *a, int v) {
	for (int i = 0; i < m; ++i) if (i < rev[i]) swap(a[i] , a[rev[i]]);
	for (int s = 2; s <= m; s <<= 1) {
		comp g(cos(2 * pi / s) , v * sin(2 * pi / s));
		// NTT: int g = power(gg, (mod - 1) / s);
		// NTT: if (v == -1) g = power(g, mod - 2);
		for (int k = 0; k < m; k += s) {
			comp w(1, 0);
			// NTT: int w = 1;
			for (int j = 0; j < s / 2; ++j) {
				comp &u = a[k + j + s / 2], &v = a[k + j];
				comp t = w * u; u = v - t; v = v + t; w = w * g;
				// NTT: be aware of "+-*"
			}
		}
	}
	if (v == -1) for (int i = 0; i < m; ++i) a[i] /= m;
	// NTT: be aware of "/"
}
\end{lstlisting}

\subsection{求原根}
\begin{lstlisting}
vector <LL> a;
bool g_test(LL g, LL p) { for (LL i = 0; i < a.size(); ++i) if (pow_mod(g, (p-1)/a[i], p) == 1) return 0; return 1; }
LL p_root(LL p) {
	LL tmp = p - 1;
	for (LL i = 2; i <= tmp / i; ++i)
		if (tmp % i == 0) { a.push_back(i); while (tmp % i == 0) tmp /= i; }
	if (tmp != 1) a.push_back(tmp);
	LL g = 1; while (1) { if (g_test(g, p)) return g; ++g; }
}
\end{lstlisting}

\subsection{FWT}
给定长度为 $2^n$ 的序列 $A[0 \cdots 2^n-1], B[0 \cdots 2^n-1]$ ,求这两序列的
\ding{172} $or$ 卷积: $C_k = \sum\limits_{i \ or \ j=k} A_iB_j$;
\ding{173} $and$ 卷积: $C_k = \sum\limits_{i \ and \ j=k} A_iB_j$;
\ding{174} $xor$ 卷积: $C_k = \sum\limits_{i \ xor \ j=k} A_iB_j$。
\begin{lstlisting}
void FWT(int *a, int n) {
    for (int d = 1; d < n; d <<= 1)
        for (int m = d << 1, i = 0; i < n; i += m)
            for (int j = 0; j < d; ++j) {
                int x = a[i + j], y = a[i + j + d];
                //or: a[i + j + d] = x + y;
                //and: a[i + j] = x + y;
                //xor: a[i + j] = x + y, a[i + j + d] = x - y;
                // 如答案要求取模,此处记得取模
            }
}
void UFWT(int *a, int n) {
    for (int d = 1; d < n; d <<= 1)
        for (int m = d << 1, i = 0; i < n; i += m)
            for (int j = 0; j < d; ++j) {
                int x = a[i + j], y = a[i + j + d];
                //or: a[i + j + d] = y - x;
                //and: a[i + j] = x - y;
                //xor: a[i + j] = (x + y) * 2^(-1), a[i + j + d] = (x - y) * 2^(-1);
                // 如答案要求取模,此处记得取模;2^(-1)表示2的逆元。
            }
}
\end{lstlisting}

\subsection{线性基}

\begin{lstlisting}
#define B 30
const int allset=(1<<B)-1;
struct LB {
	int mat[B],cnt;
	multiset<int> st;
	LB(){}
	void clear() { st.clear(); cnt=0; memset(mat,0,sizeof(mat)); }
	void add(int x) {
		for (int i=B-1;i>=0;i--) if ((x>>i)&1) {
		    if (mat[i]) x^=mat[i]; 
		    else { cnt++; mat[i]=x; break; }
		}
	}
	void fix() {
		for (int i=0;i<B;i++) if (mat[i])
		    for (int j=i+1;j<B;j++) if ((mat[j]>>i)&1) mat[j]^=mat[i];
	}
	void preset() { //正确性待定
		fix(); for (int i=0;i<B;i++) if (mat[i]) st.insert(mat[i]);
	}
	int kth(int k) { //正确性待定
		int i=0,ans=0; if (k<=0||k>(1<<cnt)-1) return 0;//无解
		for (multiset<int>::iterator it=st.begin();it!=st.end();it++,i++)
			if ((k>>i)&1) ans^=(*it);
		return ans;
	}
	int getmax() {
		fix(); int ans=0;
		for (int i=B-1;i>=0;i--) if (ans^mat[i]>ans) ans^=mat[i];
		return ans;
	}
} tree[N*10];
\end{lstlisting}

\section{计算几何}

\subsection{凸包}
\begin{lstlisting}
bool cmp(const Point &a,const Point &b) {
	return F(a.x-b.x)<0||F(a.x-b.x)==0&&a.y<b.y; }
void Gram(int id[], int n) {
	int i,mid; sort(id,id+n,cmp); tp=0; //凸包从x最小的点出发,逆时针方向
     for (i=0;i<n;i++) {
         for (;tp>=2&&Cross(p[sk[tp-1]]-p[sk[tp-2]],p[id[i]]-p[sk[tp-1]])<=0;tp--); 
		//有重点必须用<=不留共线点,无重点用<=不留共线点,无重点用<留共线点
         sk[tp++]=id[i];
     }
     mid=tp;
     for (i=n-2;i>=0;i--) {
         for (;tp>mid&&Cross(p[sk[tp-1]]-p[sk[tp-2]],p[id[i]]-p[sk[tp-1]])<=0;tp--); 
		//有重点必须用<=不留共线点,无重点用<=不留共线点,无重点用<留共线点
         sk[tp++]=id[i];
     }
     if  (n>1) tp--;
}
\end{lstlisting}

\subsection{定义}
\begin{lstlisting}
struct Point { double x,y; Point(){} Point(double _x,double _y):x(_x),y(_y){} };
struct Seg { Point a,b; Seg(){} Seg(Point _a,Point _b):a(_a),b(_b){} };
struct Circle { double x,y,r;
	Point pt() { return Point(x,y); }
	double Area() { return pi*r*r; }
};
Point operator +(const Point &a,const Point &b);
Point operator -(const Point &a,const Point &b);
Point operator *(const Point &a,double b);
Point operator /(const Point &a,double b);
int F(double x) { if (x>eps) return 1; if (x<-eps) return -1; return 0; }
bool operator ==(const Point &a,const Point &b) {
	return F(a.x-b.x)==0&&F(a.y-b.y)==0; }
double Dist(const Point &a) { return sqrt(a.x*a.x+a.y*a.y); }
double Dot(const Point &a,const Point &b) { return a.x*b.x+a.y*b.y; }
double Cross(const Point &a,const Point &b) { return a.x*b.y-a.y*b.x; }
Point Rotate(const Point &p,double a) { // 逆时针旋转
	return Point(p.x*cos(a)-p.y*sin(a),p.x*sin(a)+p.y*cos(a)); }
Point Inter(Seg a,Seg b) { // 两线段相交(前提有交点)
	double s=Cross(a.b-a.a,b.a-a.a),t=Cross(a.b-a.a,b.b-a.a);
	return b.a+(b.b-b.a)*s/(s-t); }
vector<Point> SegCir(Seg seg,Point pt,double r) { //线圆
	vector<Point> ans; double mul; Point vec,mid;
	ans.clear(); vec=Rotate(seg.b-seg.a,pi/2);
	mid=Inter(seg,Seg(pt,pt+vec));
	if (F(Dist(pt-mid)-r)>0) return ans;
	if (F(Dist(pt-mid)-r)==0) {
		ans.push_back(mid); ans.push_back(mid); return ans;
	}
	vec=seg.b-seg.a; mul=sqrt(r*r-Dist2(mid-pt))/Dist(vec);
	ans.push_back(mid+vec*mul); ans.push_back(mid-vec*mul);
	return ans;
}
vector<Point> Circir(Circle a,Circle b) { //圆圆相交
	vector<Point> ans; double dis,dis2,alpha; Point pa,pb,vec;
	ans.clear(); if (a.r<b.r) swap(a,b);
	pa=a.pt(); pb=b.pt(); vec=pb-pa;
	dis=Dist(vec); dis2=Dist2(vec);
	if (F(dis-(a.r+b.r))>0||F(dis-(a.r-b.r))<0) return ans;
	if (F(dis-(a.r+b.r))==0) {
		ans.push_back(pa+vec*a.r/(a.r+b.r)); return ans;
    }
	if (F(dis-(a.r-b.r))==0) {
		ans.push_back(pa+vec*a.r/(a.r-b.r)); return ans;
	}
	alpha=acos((a.r*a.r+dis2-b.r*b.r)/2/a.r/dis);
	ans.push_back(pa+Rotate(vec,alpha)*a.r/dis);
	ans.push_back(pa+Rotate(vec,-alpha)*a.r/dis);
	return ans;
}
double Bing(double ra,double rb,double dis) {
	double alpha,beta; if (ra<rb) swap(ra,rb);
	if (F(dis-(ra-rb))<=0) return pi*ra*ra;
	if (F(dis-(ra+rb))>=0) return pi*ra*ra+pi*rb*rb;
	alpha=acos((ra*ra+dis*dis-rb*rb)/2/dis/ra);
	beta=acos((rb*rb+dis*dis-ra*ra)/2/dis/rb);
	return (pi-alpha)*ra*ra+(pi-beta)*rb*rb+ra*dis*sin(alpha);
}
double Jiao(double ra,double rb,double dis) {
	return pi*ra*ra+pi*rb*rb-Bing(ra,rb,dis); }
Point Gongmid(Circle a,Circle b) { //正确性待定
	Point pa=a.pt(),pb=b.pt();
	return pa+(pb-pa)*a.r/(a.r+b.r); }
Point Gongright(Circle a,Circle b) {
	Point pa=a.pt(),pb=b.pt();
	return pa+(pb-pa)*a.r/(a.r-b.r); }
int Ptinpol(Point pt) {
	int wn=0;
    for(int i=0;i<n;i++) {
        if(Ins(pt,Seg(p[i],p[(i+1)%n]))) return 2;
        int k=F(Cross(p[(i+1)%n]-p[i],pt-p[i]));
        int d1=F(p[i].y-pt.y), d2=F(p[(i+1)%n].y-pt.y);
        if(k>0&&d1<=0&&d2>0)wn++;
        if(k<0&&d2<=0&&d1>0)wn--;
    }
    return wn!=0;
}
bool Cirinpol(Point pt) { //需要点在多边形内的前提
	double nearest=inf;
	for (int i=0;i<n;i++) {
		nearest=min(nearest,Dist(p[i]-pt));
		if  (F(Dot(pt-p[i],p[(i+1)%n]-p[i]))>0&&
			F(Dot(pt-p[(i+1)%n],p[i]-p[(i+1)%n]))>0)
		nearest=min(nearest,abs(Cross(p[i]-pt,p[(i+1)%n]-pt))/dis[i]);
	}
	return F(nearest-r)>=0;
}
bool Ins(const Point &p,const Seg &s) {
	return F(Cross(s.a-p,s.b-p))==0&&F(p.x-min(s.a.x,s.b.x))>=0&&
		F(p.x-max(s.a.x,s.b.x))<=0&&F(p.y-min(s.a.y,s.b.y))>=0&&
		F(p.y-max(s.a.y,s.b.y))<=0; }
double PS(const Point &p,const Seg &s) { // 点到线段最短距离
	if (F(Dot(p-s.a,s.b-s.a))<0||F(Dot(p-s.b,s.a-s.b))<0)
		return min(Dist(p-s.a),Dist(p-s.b));
	return abs(Cross(s.a-p,s.b-p))/Dist(s.a-s.b); }
double SS(const Seg &a,const Seg &b) { // 线段到线段最短距离
	return min(min(PS(a.a,b),PS(a.b,b)),min(PS(b.a,a),PS(b.b,a))); }
double Alpha(Point a,Point b) {
	double ans=atan2(b.y,b.x)-atan2(a.y,a.x);
	if (ans<0) ans=-ans; if (ans>pi) ans=2*pi-ans; return ans; }
double Shan(Circle c,double a) { return c.r*c.r*a/2; }
\end{lstlisting}

\subsection{半平面交}
\begin{lstlisting}
bool Cmphp(Seg a,Seg b) {
	Point va=a.b-a.a, vb=b.b-b.a;
    double dega=atan2(va.y,va.x), degb=atan2(vb.y,vb.x);
	return F(dega-degb)<0||F(dega-degb)==0&&Cross(a.b-a.a,b.a-a.a)<0;
}
void HalfPlane(Seg hp[], int n, Point pol[], int &pols) {
	Point mid;
	hp[n++]=Seg(Point(-oo,-oo),Point(oo,-oo));
	hp[n++]=Seg(Point(oo,-oo),Point(oo,oo));
	hp[n++]=Seg(Point(oo,oo),Point(-oo,oo));
	hp[n++]=Seg(Point(-oo,oo),Point(-oo,-oo));
	sort(hp,hp+n,Cmphp);
	int tp=0, low=0, high=-1; //sk 0~tp-1
	for (int i=0;i<n;i++)
	if (high-low+1==0||F(Cross(sk[high].b-sk[high].a,hp[i].b-hp[i].a))) {
		for (;low<high;high--) {
			mid=Inter(sk[high],sk[high-1]);
			if (F(Cross(hp[i].b-hp[i].a,mid-hp[i].a))>0) break;
		}
		for (;low<high;low++) {
			mid=Inter(sk[low],sk[low+1]);
			if (F(Cross(hp[i].b-hp[i].a,mid-hp[i].a))>0) break;
		}
		sk[++high]=hp[i];
	}
	for (;low<high;high--) {
		mid=Inter(sk[high],sk[high-1]);
		if (Cross(sk[low].b-sk[low].a,mid-sk[low].a)>0) break;
	}
	tp=high-low+1; for (int i=0;i<tp;i++) sk[i]=sk[low+i];
	pols=0; if  (tp<=2) return;
	for (int i=0;i<tp;i++) pol[pols++]=Inter(sk[i],sk[(i+1)%tp]);
}
\end{lstlisting}

\subsection{圆与多边形交集}
\begin{lstlisting}
double CT(Circle c,Point a,Point b) { // 圆与三角形交(多边形)
	double da=Dist(a-c.pt()), db=Dist(b-c.pt());
	if (da>db) { swap(a,b); swap(da,db); }
	Seg s=Seg(a,b);
	vector<Point> temp=CS(c,s);
	if (F(db-c.r)<=0) return 0.5*abs(Cross(a-c.pt(),b-c.pt()));
	if (F(da-c.r)<0) {
		if (F(Dot(a-temp[1],b-temp[1]))<0) swap(temp[0],temp[1]);
		return Shan(c,Alpha(temp[0]-c.pt(),b-c.pt()))+
			0.5*abs(Cross(a-c.pt(),temp[0]-c.pt()));
	}
	if (!temp.size()) return Shan(c,Alpha(a-c.pt(),b-c.pt()));
	if (Ins(temp[1],s)&&Dist2(a-temp[1])<Dist2(a-temp[0])) swap(temp[0],temp[1]);
	if (Ins(temp[0],s)&&Ins(temp[1],s)) {
		return Shan(c,Alpha(a-c.pt(),temp[0]-c.pt()))+
			Shan(c,Alpha(b-c.pt(),temp[1]-c.pt()))+
			0.5*abs(Cross(temp[0]-c.pt(),temp[1]-c.pt()));
	}
	return Shan(c,Alpha(a-c.pt(),b-c.pt()));
}
\end{lstlisting}

\subsection{三角形面积并}
\begin{lstlisting}
#define pr pair<ld,ld>
typedef long double ld;
const ld EPS=1e-8, INF=1e100;
struct Point {
	ld x,y; Point(){} Point(ld _,ld __):x(_),y(__){}
	void read() { double _x,_y; scanf("%lf%lf",&_x,&_y); x=_x,y=_y; }
	friend bool operator<(Point a,Point b) {
		if(fabs(a.x-b.x)<EPS) return a.y<b.y;
        return a.x<b.x;
    }
	friend Point operator +(Point a,Point b) { return Point(a.x+b.x,a.y+b.y); }
	friend Point operator -(Point a,Point b) { return Point(a.x-b.x,a.y-b.y); }
	friend Point operator *(ld a,Point b) { return Point(a*b.x,a*b.y); }
	friend ld operator *(Point a,Point b) { return a.x*b.x+a.y*b.y; }
	friend ld operator ^(Point a,Point b) { return a.x*b.y-a.y*b.x; }
} a[N][3],Poi[N*N];
struct Line {
    Point p,v; Line(){} Line(Point x,Point y){p=x,v=y-x;}
	Point operator [](int k) { if(k) return p+v; else return p; }
	friend bool Cross(Line a,Line b) {
		return (a.v^b[0]-a.p)*(a.v^b[1]-a.p)<-EPS &&
			(b.v^a[0]-b.p)*(b.v^a[1]-b.p)<-EPS;
	}
	friend Point getP(Line a,Line b) {
		Point u=a.p-b.p; ld temp=(b.v^u)/(a.v^b.v);
		return a.p+temp*a.v;
	}
}l[N][3],T;
pr p[N];
int main() {
	int n,m,i,j,k,x,y,cnt,tot;
	ld ans,last,A,B,sum;
	scanf("%d",&n);
	for(i=1,tot=0;i<=n;i++) {
		a[i][0].read(),a[i][1].read(),a[i][2].read();
		Poi[++tot]=a[i][0],Poi[++tot]=a[i][1],Poi[++tot]=a[i][2];
		sort(a[i],a[i]+3);
		if((a[i][2]-a[i][0]^a[i][1]-a[i][0])>EPS)
			l[i][0]=Line(a[i][0],a[i][2]),l[i][1]=Line(a[i][2],a[i][1]),l[i][2]=Line(a[i][1],a[i][0]);
		else
			l[i][0]=Line(a[i][2],a[i][0]),l[i][1]=Line(a[i][1],a[i][2]),l[i][2]=Line(a[i][0],a[i][1]);
    }
	for(i=1;i<=n;i++) for(j=1;j<i;j++) for(x=0;x<3;x++) for(y=0;y<3;y++)
		if(Cross(l[i][x],l[j][y])) Poi[++tot]=getP(l[i][x],l[j][y]);
	sort(Poi+1,Poi+tot+1);
	ans=0,last=Poi[1].x; T=Line(Point(0,-INF),Point(0,INF));
	for(i=2;i<=tot;i++) {
        T.p.x=(last+Poi[i].x)/2;
        for(j=1,cnt=0;j<=n;j++)
            if(Cross(l[j][0],T)) {
                if(Cross(l[j][1],T)) B=getP(l[j][1],T).y;
                else B=getP(l[j][2],T).y;
                A=getP(l[j][0],T).y; if (A>B) swap(A,B);
                p[++cnt]=pr(A,B);
            }
        sort(p+1,p+cnt+1);
        for(j=1,sum=0,A=-INF;j<=cnt;j++) {
            if(p[j].first>A) sum+=p[j].second-p[j].first, A=p[j].second;
            else if(p[j].second>A) sum+=p[j].second-A, A=p[j].second;
        }
		ans+=(Poi[i].x-last)*sum; last=Poi[i].x;
	}
	printf("%.2lf\n",(double)ans);
}
\end{lstlisting}

\subsection{K圆并}
\begin{lstlisting}
#define sqr(x) ((x)*(x))
const double eps = 1e-8;
double area[N]; int n;
int dcmp(double x) { if (x < -eps) return -1; else return x > eps; }
struct cp { double x, y, r, angle; int d;
	cp(){} cp(double xx, double yy, double ang = 0, int t = 0) {
		x = xx;  y = yy;  angle = ang;  d = t; }
	void get() { scanf("%lf%lf%lf", &x, &y, &r); d = 1; }
}cir[N], tp[N * 2];
double dis(cp a, cp b) { return sqrt(sqr(a.x - b.x) + sqr(a.y - b.y)); }
double cross(cp p0, cp p1, cp p2) {
	return (p1.x - p0.x) * (p2.y - p0.y) - (p1.y - p0.y) * (p2.x - p0.x); 
}
int CirCrossCir(cp p1, double r1, cp p2, double r2, cp &cp1, cp &cp2) {
    double mx = p2.x - p1.x, sx = p2.x + p1.x, mx2 = mx * mx;
    double my = p2.y - p1.y, sy = p2.y + p1.y, my2 = my * my;
    double sq = mx2 + my2, d = -(sq - sqr(r1 - r2)) * (sq - sqr(r1 + r2));
    if (d + eps < 0) return 0; if (d < eps) d = 0; else d = sqrt(d);
    double x = mx * ((r1 + r2) * (r1 - r2) + mx * sx) + sx * my2;
    double y = my * ((r1 + r2) * (r1 - r2) + my * sy) + sy * mx2;
    double dx = mx * d, dy = my * d; sq *= 2;
    cp1.x = (x - dy) / sq; cp1.y = (y + dx) / sq;
    cp2.x = (x + dy) / sq; cp2.y = (y - dx) / sq;
    if (d > eps) return 2; else return 1;
}
bool circmp(const cp& u, const cp& v) { return dcmp(u.r - v.r) < 0; }
bool cmp(const cp& u, const cp& v) {
    if (dcmp(u.angle - v.angle)) return u.angle < v.angle;
    return u.d > v.d;
}
double calc(cp cir, cp cp1, cp cp2) {
    double ans = (cp2.angle - cp1.angle) * sqr(cir.r)
        - cross(cir, cp1, cp2) + cross(cp(0, 0), cp1, cp2);
    return ans / 2;
}
void CirUnion(cp cir[], int n) {
    cp cp1, cp2; sort(cir, cir + n, circmp);
    for (int i = 0; i < n; ++i) for (int j = i + 1; j < n; ++j)
        if (dcmp(dis(cir[i], cir[j]) + cir[i].r - cir[j].r) <= 0) cir[i].d++;
	for (int i = 0; i < n; ++i) {
		int tn = 0, cnt = 0;
		for (int j = 0; j < n; ++j) {
			if (i == j) continue;
            if (CirCrossCir(cir[i],cir[i].r,cir[j],cir[j].r,cp2,cp1)<2) continue;
            cp1.angle = atan2(cp1.y - cir[i].y, cp1.x - cir[i].x);
            cp2.angle = atan2(cp2.y - cir[i].y, cp2.x - cir[i].x);
            cp1.d = 1; tp[tn++] = cp1; cp2.d = -1; tp[tn++] = cp2;
            if (dcmp(cp1.angle - cp2.angle) > 0) cnt++;
        }
        tp[tn++] = cp(cir[i].x - cir[i].r, cir[i].y, pi, -cnt);
        tp[tn++] = cp(cir[i].x - cir[i].r, cir[i].y, -pi, cnt);
        sort(tp, tp + tn, cmp);
        int p, s = cir[i].d + tp[0].d;
        for (int j = 1; j < tn; ++j) {
            p = s;  s += tp[j].d;
            area[p] += calc(cir[i], tp[j - 1], tp[j]);
        }
    }
}
void solve() {
    for (int i = 0; i < n; ++i) cir[i].get();
    memset(area, 0, sizeof(area));
    CirUnion(cir, n);
    for (int i = 1; i <= n; ++i) {
        area[i] -= area[i + 1];
        printf("[%d] = %.3lf\n", i, area[i]);
    }
}
\end{lstlisting}

\subsection{三维计算几何}
\begin{lstlisting}
Point Cross(Point a,Point b) {
	return Point(a.y*b.z-a.z*b.y,a.z*b.x-a.x*b.z,a.x*b.y-a.y*b.x); }
double Crossxy(Point a,Point b) { return a.x*b.y-a.y*b.x; }
vector<Point> SegPlane(Seg seg,Plane p) {
	vector<Point> ans; ans.clear();
	Point fa=Cross(p.b-p.a,p.c-p.a);
	if  (F(Dot(fa,seg.b-seg.a))==0) return ans;
	double s=Dot(p.a-seg.a,fa)/Dist(fa), t=Dot(p.a-seg.b,fa)/Dist(fa);
	ans.push_back(seg.a+(seg.b-seg.a)*s/(s-t));
	return ans;
}
// mixed product
double Mix(Point3 a,Point3 b,Point3 c) { return Dot(Cross(a,b),c); }
double PP(Point3 pt,Plane pl) { // distance from point to plane
	Point3 fa=Cross(pl.b-pl.a,pl.c-pl.a);
	return abs(Dot(fa,pt-pl.a))/Dist(fa);
}
// get the center point from 3D(need plane well prepared)
Point3 Getcenter(Point3 p[],int n,Plane pp[],int nn) {
	double sumv=0;
	Point3 sum=Point3(0,0,0);
	for (int i=0;i<nn;i++)
	{
		double tempv=Mix(pp[i].b-pp[i].a,pp[i].c-pp[i].a,Point3(0,0,0)-pp[i].a);
		sum=sum+(pp[i].a+pp[i].b+pp[i].c)*tempv/4.0;
		sumv+=tempv;
	}
	return sum/sumv;
}
\end{lstlisting}


\section{字符串}
\subsection{哈希}
\begin{lstlisting}
const int P=31,D=1000173169;
int hash(int l, int r) { return (LL)(f[r]-(LL)f[l-1]*pow[r-l+1]%D+D)%D; }
pow[0] = 1;	for (int i=1;i<=n;i++) pow[i] = (LL)pow[i-1]*P%D;
for (int i=1;i<=n;i++) f[i] = (LL)((LL)f[i-1]*P+a[i])%D;
\end{lstlisting}
\subsection{KMP}
输入:模式串长度 $n$ ,模式串 $a$ ,匹配串长度 $m$ ,匹配串 $b$;输出: 依次输出每个匹配成功的\emph{起始}位置;下标从 $0$ 开始。

\begin{lstlisting}
void kmp(int n, char* a, int m, char *b) {
    int i, j;
    for (nxt[0] = j = -1, i = 1; i < n; nxt[i++] = j) {
        while (~j && a[j + 1] != a[i]) j = nxt[j];
        if (a[j + 1] == a[i]) ++j;
    }
    for (j = -1, i = 0; i < m; ++i) {
        while (~j && a[j + 1] != b[i]) j = nxt[j];
        if (a[j + 1] == b[i]) ++j;
        if (j == n - 1) {
            printf("%d\n", i - n + 1);
            j = nxt[j];
        }
    }
}
\end{lstlisting}

\subsection{扩展KMP}

next:  a 关于自己每个后缀的最长公共前缀;ret: a 关于 b 的每个后缀的最长公共前缀;EXKMP的next[i]表示:从i到n-1的字符串st前缀和原串前缀的最长重叠长度。

\begin{lstlisting}
void get_next(char *a, int *next) {
	int i, j, k, n = strlen(a);
	for (j = 0; j+1<n && a[j]==a[j+1];j++);
	next[1] = j; k = 1;
	for (i=2;i<n;i++) {
		int len = k+next[k], L = next[i-k];
		if (L < len-i) {
			next[i] = L;
		} else {
			for (j = max(0, len-i);i+j<n && a[j]==a[i+j];j++);
			next[i] = j;
			k = i;
		}
	}
}
void ExtendedKMP(char *a, char *b, int *next, int *ret) {
	get_next(a, next);
	int n = strlen(a), m = strlen(b);
	int i, j, k;
	for (j=0;j<n && j<m && a[j]==b[j];j++);
	ret[0] = j;
	k = 0;
	for (i=1;i<m;i++) {
		int len = k+ret[k], L = next[i-k];
		if (L < len-i) {
			ret[i] = L;
		} else {
			for (j = max(0, len-i);j<n && i+j<m && a[j]==b[i+j];j++);
			ret[i] = j;
			k = i;
		}
	}
}
\end{lstlisting}

\subsection{Manacher}
p[i] 表示以 i 为对称轴的最长回文串长度
\begin{lstlisting}
char st[N*2], s[N];
int len, p[N*2];
while (scanf("%s", s) != EOF) {
	len = strlen(s);
	st[0] = '$', st[1] = '#';
	for (int i=1;i<=len;i++)
		st[i*2] = s[i-1], st[i*2+1] = '#';
	len = len * 2 + 2;
	int mx = 0, id = 0, ans = 0;
	for (int i=1;i<=len;i++) {
		p[i] = (mx > i) ? min(p[id*2-i]+1, mx-i) : 1;
		for (; st[i+p[i]] == st[i-p[i]]; ++p[i]) ;
		if (p[i]+i > mx) mx = p[i]+i, id = i;
		p[i] --;
		if (p[i] > ans) ans = p[i];
	}
	printf("%d\n", ans);
}
\end{lstlisting}

\subsection{AC自动机}
\begin{lstlisting}
struct Node { int next[26]; int terminal, fail; };
void build() {
	head = 0, tail = 1; q[1] = 1;
	while (head != tail) {
		int x = q[++head];
		/*(when necessary) node[x].terminal |= node[node[x].fail].terminal; */
		for (int i=0;i<26;i++)
			if (node[x].next[i]) {
				int y = node[x].fail;
				while (y) {
					if (node[y].next[i]) {
						node[node[x].next[i]].fail = node[y].next[i];
						break;
					}
					y = node[y].fail;
				}
				if (!node[node[x].next[i]].fail) node[node[x].next[i]].fail = 1;
				q[++tail] = node[x].next[i];
			}
	}
}
\end{lstlisting}

\subsection{后缀数组}
参数 m 表示字符集的大小,即 $0 \leq r_i < m$ 
\begin{lstlisting}
int n, r[N], wa[N], wb[N], ws[N], sa[N], rank[N], height[N];
int cmp(int *r, int a, int b, int l, int n) { return r[a]==r[b] && a+l<n && b+l<n && r[a+l]==r[b+l]; }
void suffix_array(int m) {
	int i, j, p, *x=wa, *y=wb, *t;
	for (i=0;i<m;i++) ws[i]=0; for (i=0;i<n;i++) ws[x[i]=r[i]]++;
	for (i=1;i<m;i++) ws[i]+=ws[i-1]; for (i=n-1;i>=0;i--) sa[--ws[x[i]]]=i;
	for (j=1,p=1;p<n;m=p,j<<=1) {
		for (p=0,i=n-j;i<n;i++) y[p++]=i;
		for (i=0;i<n;i++) if (sa[i]>=j) y[p++]=sa[i]-j;
		for (i=0;i<m;i++) ws[i]=0; for (i=0;i<n;i++) ws[x[y[i]]]++;
		for (i=1;i<m;i++) ws[i]+=ws[i-1];
		for (i=n-1;i>=0;i--) sa[--ws[x[y[i]]]]=y[i];
		for (t=x,x=y,y=t,x[sa[0]]=0,i=1,p=1;i<n;i++)
			x[sa[i]]=cmp(y,sa[i-1],sa[i],j,n)?p-1:p++;
	}
	for (i=0;i<n;i++) rank[sa[i]]=i; rank[n] = -1;
	for (i=j=0;i<n;i++) if (rank[i]) {
		while (r[i+j]==r[sa[rank[i]-1]+j]) j++; 
		height[rank[i]]=j;
		if (j) j--; 
	}
}
\end{lstlisting}
\subsection{后缀自动机}
下面的代码是求两个串的LCS(最长公共子串)。
\begin{lstlisting}
#define M (N << 1)
char st[N];
int pre[M], son[26][M], step[M], refer[M], size[M], tmp[M], topo[M], last, total;
int apply(int x, int now) {
	step[++total] = x;
	refer[total] = now;
	return total;
}
void extend(int x, int now) {
	int p = last, np = apply(step[last]+1, now);
	size[np] = 1;
	for (; p && !son[x][p]; p=pre[p]) son[x][p] = np;
	if (!p) pre[np] = 1;
	else {
		int q = son[x][p];
		if (step[p]+1 == step[q]) pre[np] = q;
		else {
			int nq = apply(step[p]+1, now);
			for (int i=0;i<26;i++) son[i][nq] = son[i][q];
			pre[nq] = pre[q]; pre[q] = pre[np] = nq;
			for (; p && son[x][p]==q; p=pre[p]) son[x][p] = nq;
		}
	}
	last = np;
}
void init() {
	last = total = 0;
	last = apply(0, 0);
	scanf("%s",st);
	int n = strlen(st);
	for (int i = 0; i <= n * 2; ++i) {
		pre[i] = step[i] = refer[i] = size[i] = tmp[i] = topo[i] = 0;
		for (int j = 0; j < 26; ++j) son[j][i] = 0;
	}
	for (int i = 0; i < n; ++i) extend(st[i] - 'a', i);
	for (int i = 1; i <= total; ++i) tmp[step[i]] ++;
	for (int i = 1; i <= n; ++i) tmp[i] += tmp[i - 1];
	for (int i = 1; i <= total; ++i) topo[tmp[step[i]]--] = i;
	for (int i = total; i; --i) size[pre[topo[i]]] += size[topo[i]];
}
int main() {
	init();
	int p = 1, now = 0, ans = 0;
	scanf("%s", st);
	for (int i=0; st[i]; i++) {
		int index = st[i]-'a';
		for (; p && !son[index][p]; p = pre[p], now = step[p]) ;
		if (!p) p = 1;
		if (son[index][p]) {
			p = son[index][p]; now++;
			if (now > ans) ans = now;
		}
	}
	printf("%d\n",ans);
	return 0;
}
\end{lstlisting}

一些定义和性质:
\ding{172} Right(str) 表示 str 在母串 S 中所有出现的结束位置集合;
\ding{173} 一个状态 s 表示的所有子串 Right 集合相同,为 Right(s);
\ding{174} Parent(s) 满足 Right(s) 是 Right(Parent(s)) 的真子集,并且 Right(Parent(s)) 的大小最小;
\ding{175} Parent 函数可以表示一个树形结构。不妨叫它 Parent 树;
\ding{176} 一个 Right 集合和一个长度定义了一个子串;
\ding{177} 对于状态 s ,使得 Right(s) 合法的子串长度是一个区间 [min(s), max(s)] ;
\ding{178} max(Parent(s)) = min(s) - 1;
\ding{179} 令 refer(s) 表示产生 s 状态的字符所在位置。则 Right(s) 的合法子串的起始位置为 [refer(s) - max(s) + 1, refer(s) - min(s) + 1] ,即 [refer(s) - max(s) + 1, refer(s) - max(Parent(s))] 。

代码中变量名含义:
\ding{172} pre[s] 为上述定义中的 Parent(s);
\ding{173} step[s] 为从初始状态走到 s 状态最多需要多少步;
\ding{174} refer[s] 为上述定义中的 refer(s);
\ding{175} size[s] 为 Right(s) 集合的大小;
\ding{176} topo[s] 为 Parent 树的拓扑序,根(初始状态)在前。

\subsection{回文树}

\ding{172} len[i]表示编号为i的节点表示的回文串的长度(一个节点表示一个回文串)
\ding{173} next[i][c]表示编号为i的节点表示的回文串在两边添加字符c以后变成的回文串的编号(和字典树类似)。
\ding{174} fail[i]表示节点i失配以后跳转不等于自身的节点i表示的回文串的最长后缀回文串(和AC自动机类似)。
\ding{175} cnt[i]表示节点i表示的本质不同的串的个数(建树时求出的不是完全的,最后count()函数跑一遍以后才是正确的)
\ding{176} num[i]表示以节点i表示的最长回文串的最右端点为回文串结尾的回文串个数。
\ding{177} last指向新添加一个字母后所形成的最长回文串表示的节点。
\ding{178} st[i]表示第i次添加的字符(一开始设st[0] = -1(可以是任意一个在串S中不会出现的字符))。
\ding{179} tot 表示添加的节点个数。
\ding{180} n 表示添加的字符个数。

【URAL2040】Palindromes and Super Abilities 2

逐个添加字符串S里的字符 $S_1, S_2, ..., S_n$ 。每次添加字符后,他想知道添加字符后将出现多少个新的本质不同的回文子串。字符集为 $\{a, b\}$

\begin{lstlisting}
struct PAM {
	int n, tot, last, len[N], fail[N], next[N][2], num[N], cnt[N];
	void init() { n=0; tot=1; len[1]=-1; fail[1]=0; len[0]=+0; fail[0]=1; last=1; }
	int get_fail(int x) { for (; st[n-len[x]-1]!=st[n]; x=fail[x]); return x; }
	void insert(char c) {
		++n; int cur=get_fail(last); // 判断上一个串的前一个位置和新添加的位置是否相同,相同则说明构成回文。否则找 fail 指针。
		if (!next[cur][c]) {
			++tot; len[tot]=len[cur]+2; fail[tot]=next[get_fail(fail[cur])][c];
			next[cur][c]=tot; num[tot] = num[fail[tot]] + 1; answer[n]='1';
		} else answer[n]='0';
		last=next[cur][c]; cnt[last] ++;
	}
	void count () { for (int i=tot-1; i>=0; --i) cnt[fail[i]] += cnt[i]; }
	//父亲累加儿子的cnt,因为如果fail[v]=u,则u一定是v的子回文串。
} pam;
n=strlen(st+1); pam.init();
for (int i=1;i<=n;i++) pam.insert(st[i]-'a');
\end{lstlisting}

\section{数据结构}
\subsection{ST表}
\begin{lstlisting}
int Log[N],f[17][N];
int ask(int x,int y){ int k=Log[y-x+1]; return max(f[k][x],f[k][y-(1<<k)+1]); }
for (int i=2;i<=n;i++)Log[i]=Log[i>>1]+1; for (int j=1;j<K;j++) for (int i=1;i+(1<<j-1)<=n;i++) f[j][i]=max(f[j-1][i],f[j-1][i+(1<<j-1)]);
\end{lstlisting}

\subsection{K-D Tree}

\ding{172} change 将编号为 x 的点的权值增加 p;
\ding{173} euclid\_lower\_bound 欧几里得距离的平方,下界;
\ding{174} euclid\_upper\_bound 欧几里得距离的平方,上界;
\ding{175} manhattan\_lower\_bound 曼哈顿距离,下界;
\ding{176} manhattan\_upper\_bound 曼哈顿距离,上界;
\ding{177} add 添加一个点(注意此处的添加可能导致这棵树不平衡,慎用!);
\ding{178} ask(p, X, Y, ans) 询问距离点 (X, Y) 最远的一个点的距离, ans 需传入无穷小;
\ding{179} ask(p, x1, y1, x2, y2) 查询矩形范围内所有点的权值和。

\begin{lstlisting}
int n, cmp_d, root, id[N];
struct node { int d[2], l, r, Max[2], Min[2], val, sum, f; } t[N];
inline bool cmp(const node &a, const node &b) {
	if (a.d[cmp_d] != b.d[cmp_d]) return a.d[cmp_d] < b.d[cmp_d];
	return a.d[cmp_d ^ 1] < b.d[cmp_d ^ 1];
}
inline void umax(int &a, int b) { if (b > a) a = b; }
inline void umin(int &a, int b) { if (b < a) a = b; }
inline void up(int x, int y) { umax(t[x].Max[0], t[y].Max[0]); umin(t[x].Min[0], t[y].Min[0]); umax(t[x].Max[1], t[y].Max[1]); umin(t[x].Min[1], t[y].Min[1]); }
int build(int l, int r, int D, int f) {
	int mid = (l + r) / 2; cmp_d = D;
	nth_element(t + l + 1, t + mid + 1, t + r + 1, cmp);
	id[t[mid].f] = mid; t[mid].f = f;
	t[mid].Max[0] = t[mid].Min[0] = t[mid].d[0];
	t[mid].Max[1] = t[mid].Min[1] = t[mid].d[1];
	t[mid].val = t[mid].sum = 0;
	if (l != mid) t[mid].l = build(l, mid - 1, !D, mid);
	else t[mid].l = 0;
	if (r != mid) t[mid].r = build(mid + 1, r, !D, mid);
	else t[mid].r = 0;
	if (t[mid].l) up(mid, t[mid].l);
	if (t[mid].r) up(mid, t[mid].r);
	return mid;
}
void change(int x, int p) {
	x = id[x]; // 将点的编号映成排序后的编号
	for (t[x].val += p; x; x = t[x].f) t[x].sum += p;
}
inline long long sqr(long long x) { return x * x; }
inline long long euclid_lower_bound(const node &a, int X, int Y) {
	return sqr(max(max(X - a.Max[0], a.Min[0] - X), 0)) +
		sqr(max(max(Y - a.Max[1], a.Min[1] - Y), 0)); }
inline long long euclid_upper_bound(const node &a, int X, int Y) {
	return max(sqr(X - a.Min[0]), sqr(X - a.Max[0])) +
		max(sqr(Y - a.Min[1]), sqr(Y - a.Max[1])); }
inline long long manhattan_lower_bound(const node &a, int X, int Y) {
	return max(a.Min[0] - X, 0) + max(X - a.Max[0], 0) + 
		max(a.Min[1] - Y, 0) + max(Y - a.Max[1], 0);
}
inline long long manhattan_upper_bound(const node &a, int X, int Y) {
	return max(abs(X - a.Max[0]), abs(a.Min[0] - X)) + 
		max(abs(Y - a.Max[1]), abs(a.Min[1] - Y));
}
void add(int k) {
	t[k].Max[0] = t[k].Min[0] = t[k].d[0]; t[k].Max[1] = t[k].Min[1] = t[k].d[1];
	t[k].val = t[k].sum = 0; t[k].l = t[k].r = t[k].f = 0;
	if (!root) root = k, return;
	int p = root, D = 0;
	while (1) { up(p, k);
		if (t[k].d[D] <= t[p].d[D]) { if (t[p].l) p = t[p].l; else t[p].l = k, t[k].f = p, return; }
		else { if (t[p].r) p = t[p].r; else t[p].r = k, t[k].f = p, return; } 
		D ^= 1;
	}
}
inline long long getdis(const node &a, int X, int Y) { return sqr(a.d[0] - X) + sqr(a.d[1] - Y); }
void ask(int p, int X, int Y, long long &ans) {
	if (!p) return; ans = max(ans, getdis(t[p], X, Y));
	long long dl = t[p].l ? euclid_upper_bound(t[t[p].l], X, Y) : 0;
	long long dr = t[p].r ? euclid_upper_bound(t[t[p].r], X, Y) : 0;
	if (dl > dr) { if (dl > ans) ask(t[p].l, X, Y, ans); if (dr > ans) ask(t[p].r, X, Y, ans); }
	else { if (dr > ans) ask(t[p].r, X, Y, ans); if (dl > ans) ask(t[p].l, X, Y, ans); }
}
int ask(int p, int x1, int y1, int x2, int y2) {
	if (t[p].Min[0] > x2 || t[p].Max[0] < x1 || t[p].Min[1] > y2 || t[p].Max[1] < y1) return 0;
	if (t[p].Min[0] >= x1 && t[p].Max[0] <= x2 && t[p].Min[1] >= y1 && t[p].Max[1] <= y2) return t[p].sum;
	int s = 0;
	if (t[p].d[0] >= x1 && t[p].d[0] <= x2 && t[p].d[1] >= y1 && t[p].d[1] <= y2) s += t[p].val;
	if (t[p].l) s += ask(t[p].l, x1, y1, x2, y2);
	if (t[p].r) s += ask(t[p].r, x1, y1, x2, y2);
	return s;
}
for (int i = 1; i <= n; ++i) t[i].d[0] = x, t[i].d[1] = y;
root = build(1, n, 0, 0);
\end{lstlisting}

\subsection{左偏树}

左偏树是一个可并堆。下面的程序写的是一个小根堆,如果需要改成大根堆请在注释了 here 那行修改。
接口:
\ding{172} push 插入一个元素;
\ding{173} merge 合并两个堆,注意,合并后原来那个堆将不可访问;
\ding{174} top 返回堆顶元素;
\ding{175} pop 删除堆顶元素;
\ding{176} size 返回堆的大小。

\begin{lstlisting}
template <class T> class leftist { public:
	struct node { T key; int dist; node *l, *r; };
	leftist() : root(NULL), s(0) {}
	void push(const T &x) { leftist y; y.s = 1; y.root = new node; y.root -> key = x; y.root -> dist = 0; y.root -> l = y.root -> r = NULL; merge(y); }
	node* merge(node *x, node *y) {
		if (x == NULL) return y; if (y == NULL) return x;
		if (y -> key < x -> key) swap(x, y); //here
		x -> r = merge(x -> r, y);
		int ld = x -> l ? x -> l -> dist : -1;
		int rd = x -> r ? x -> r -> dist : -1;
		if (ld < rd) swap(x -> l, x -> r);
		if (x -> r == NULL) x -> dist = 0;
		else x -> dist = x -> r -> dist + 1; return x;
	}
	void merge(leftist &x) { root = merge(root, x.root); s += x.s; }
	T top() const { if (root == NULL) return T(); return root -> key; }
	void pop() { if (root == NULL) return; node *p = root; root = merge(root -> l, root -> r); --s; delete p; }
	int size() const { return s; }
private: node* root; int s;
};
\end{lstlisting}
\subsection{线段树小技巧}
给定一个序列 $a$ ,寻找一个最大的 $i$ 使得 $i \leq y$ 且满足一些条件(如 $a[i] \geq w$ ,那么需要在线段树维护 $a$ 的区间最大值)
\begin{lstlisting}
int queryl(int p, int left, int right, int y, int w) {
	if (right <= y) {
		if (! __condition__ ) return -1;
		else if (left == right) return left;
	}
	int mid = (left + right) / 2;
	if (y <= mid) return queryl(p<<1|0, left, mid, y, w);
	int ret = queryl(p<<1|1, mid+1, right, y, w);
	if (ret != -1) return ret;
	return queryl(p<<1|0, left, mid, y, w);
}
\end{lstlisting}
给定一个序列 $a$ ,寻找一个最小的 $i$ 使得 $i \geq x$ 且满足一些条件(如 $a[i] \geq w$ ,那么需要在线段树维护 $a$ 的区间最大值)
\begin{lstlisting}
int queryr(int p, int left, int right, int x, int w) {
	if (left >= x) {
		if (! __condition__ ) return -1;
		else if (left == right) return left;
	}
	int mid = (left + right) / 2;
	if (x > mid) return queryr(p<<1|1, mid+1, right, x, w);
	int ret = queryr(p<<1|0, left, mid, x, w);
	if (ret != -1) return ret;
	return queryr(p<<1|1, mid+1, right, x, w);
}
\end{lstlisting}
\subsection{Splay}

接口: 
\ding{172} $\text{ADD x y d}$ 将 $[x, y]$ 的所有数加上 $d$ ;
\ding{173} $\text{REVERSE x y}$  将 $[x, y]$ 翻转;
\ding{174} $\text{INSERT x p}$ 将 $p$ 插入到第 $x$ 个数的后面;
\ding{175} $\text{DEL x}$ 将第 $x$ 个数删除。

\begin{lstlisting}
int w[N], Min[N], son[N][2], size[N], father[N], rev[N], lazy[N];
int top, rt, q[N];
void pushdown(int x) {
	if (!x) return;
	if (rev[x]) rev[son[x][0]] ^= 1, rev[son[x][1]] ^= 1, swap(son[x][0], son[x][1]), rev[x] = 0;
	if (lazy[x]) lazy[son[x][0]] += lazy[x], lazy[son[x][1]] += lazy[x], w[x] += lazy[x], Min[x] += lazy[x], lazy[x] = 0;
}
void pushup(int x) {
	if (!x) return; pushdown(son[x][0]); pushdown(son[x][1]);
	size[x] = size[son[x][0]] + size[son[x][1]] + 1; Min[x] = w[x];
	if (son[x][0]) Min[x] = min(Min[x], Min[son[x][0]]);
	if (son[x][1]) Min[x] = min(Min[x], Min[son[x][1]]);
}
void sc(int x, int y, int w) { son[x][w] = y; father[y] = x; pushup(x); }
void _ins(int w) {
	top++; w[top] = Min[top] = w; son[top][0] = son[top][1] = 0;
	size[top] = 1; father[top] = 0; rev[top] = 0;
}
void init() { top = 0; _ins(0); _ins(0); rt=1; sc(1, 2, 1); }
void rotate(int x) {
	if (!x) return; int y = father[x], w = son[y][1]==x;
	sc(y, son[x][w^1], w); sc(father[y], x, son[father[y]][1]==y); sc(x, y, w^1);
}
void flushdown(int x) {
	int t=0; for (; x; x=father[x]) q[++t]=x;
	for (; t; t--) pushdown(q[t]);
}
void Splay(int x, int root=0) {
	flushdown(x);
	while (father[x] != root) { int y=father[x], w=son[y][1]==x;
		if (father[y] != root && son[father[y]][w]==y) rotate(y);
		rotate(x); }
}
int find(int k) {
	Splay(rt);
	while (1) {	pushdown(rt);
		if (size[son[rt][0]]+1==k) Splay(rt), return rt;
		else if (size[son[rt][0]]+1<k) k-=size[son[rt][0]]+1, rt=son[rt][1];
		else rt=son[rt][0]; }
}
int split(int x, int y) {
	int fx = find(x), fy = find(y+2); Splay(fx); Splay(fy, fx); return son[fy][0]; }
void add(int x, int y, int d) { //add d to each number in a[x]...a[y]
	int t = split(x, y); lazy[t] += d; Splay(t); rt=t; }
void reverse(int x, int y) { // reverse the x-th to y-th elements
	int t = split(x, y); rev[t] ^= 1; Splay(t); rt=t; }
void insert(int x, int p) { // insert p after the x-th element
	int fx = find(x+1), fy = find(x+2);
	Splay(fx); Splay(fy, fx); _ins(p); sc(fy, top, 0); Splay(top); rt=top; }
void del(int x) { // delete the x-th element in Splay
	int fx = find(x), fy = find(x+2);
	Splay(fx); Splay(fy, fx); son[fy][0] = 0; Splay(fy); rt=fy; }
\end{lstlisting}

\subsection{可持久化Treap}
接口: 
\ding{172} insert 在当前第 $x$ 个字符后插入 $c$ ;
\ding{173} del 删除第 $x$ 个字符到第 $y$ 个字符 ;
\ding{174} copy 复制第 $l$ 个字符到第 $r$ 个字符,然后粘贴到第 $x$ 个字符后 ;
\ding{175} reverse 翻转第 $x$ 个到第 $y$ 个字符 ;
\ding{176} query 表示询问当前第 $x$ 个字符是什么 。

\begin{lstlisting}
char key[N];
bool rev[N];
int lc[N], rc[N], size[N]; // if size is long long, remember here
int n, root;
LL Rand() { return rd = (rd * 20372052ll + 25022087ll) % mod; }
void init() { n = root = 0; }
inline int copy(int x) { ++ n; key[n] = key[x]; (copy rev, lc, rc, size); return n; }
inline void pushdown(int x) {
	if (!rev[x]) return;
	if (lc[x]) lc[x] = copy(lc[x]); if (rc[x]) rc[x] = copy(rc[x]);
	swap(lc[x], rc[x]); rev[lc[x]] ^= 1; rev[rc[x]] ^= 1; rev[x] = 0;
}
inline void pushup(int x) { size[x] = size[lc[x]] + size[rc[x]] + 1; }
int merge(int u, int v) {
	if (!u || !v) return u+v; pushdown(u); pushdown(v);
	int t = Rand() % (size[u] + size[v]), r;  // if size is long long, remember here
	if (t < size[u]) r = copy(u), rc[r] = merge(rc[u], v);
	else r = copy(v), lc[r] = merge(u, lc[v]);
	pushup(r); return r;
}
int split(int u, int x, int y) { // if size is long long, remember here
	if (x > y) return 0; pushdown(u);
	if (x == 1 && y == size[u]) return copy(u);
	if (y <= size[lc[u]]) return split(lc[u], x, y);
	int t = size[lc[u]] + 1; // if size is long long, remember here
	if (x > t) return split(rc[u], x-t, y-t);
	int num = copy(u); lc[num]=split(lc[u], x, t-1); rc[num]=split(rc[u], 1, y-t);
	pushup(num); return num;
}
void insert(int x, char c) {
	int t1 = split(root, 1, x), t2 = split(root, x+1, size[root]);
	key[++n] = c; lc[n] = rc[n] = rev[n] = 0; pushup(n); root = merge(merge(t1, n), t2); }
void del(int x, int y) {
	int t1 = split(root, 1, x-1), t2 = split(root, y+1, size[root]); root = merge(t1, t2); }
void copy(int l, int r, int x) {
	int t1 = split(root, 1, x), t2 = split(root, l, r), t3 = split(root, x+1, size[root]);
	root = merge(merge(t1, t2), t3); }
void reverse(int x, int y) {
	int t1 = split(root, 1, x-1), t2 = split(root, x, y), t3 = split(root, y+1, size[root]);
	rev[t2] ^= 1; root = merge(merge(t1, t2), t3); }
char query(int k) {
	int x = root;
	while (1) {	pushdown(x);
		if (k <= size[lc[x]]) x = lc[x];
		else if (k == size[lc[x]] + 1) return key[x];
		else k -= size[lc[x]] + 1, x = rc[x]; }
}
\end{lstlisting}
\subsection{可持久化并查集}

接口:
\ding{172} merge 在 time 时刻将 x 和 y 连一条边,注意加边顺序必须按 time 从小到大加边
\ding{173} GetFather 询问 time 时刻及以前的连边状态中,x所属的集合

\begin{lstlisting}
const int inf = 0x3f3f3f3f;
int father[N], Father[N], Time[N];
vector<int> e[N];
void init() { for (int i=1;i<=n;i++) father[i]=Father[i]=i,Time[i]=inf,e[i].clear(),e[i].push_back(i);}
int getfather(int x) { return (father[x]==x) ? x : father[x]=getfather(father[x]); }
int GetFather(int x, int time) {return (Time[x]<=time)?GetFather(Father[x],time):x;}
void merge(int x, int y, int time) {
	int fx = getfather(x), fy = getfather(y); if (fx == fy) return;
	if (e[fx].size() > e[fy].size()) swap(fx, fy);
	father[fx] = fy; Father[fx] = fy; Time[fx] = time;
	for (int i=0;i<e[fx].size();i++) e[fy].push_back(e[fx][i]);
}
\end{lstlisting}

\subsection{普通莫队}

分块块数为 $\sqrt{n}$ 是最优的。记每次进行 add() 操作的复杂度为 $O(A)$ ,del() 操作的复杂度为 $O(D)$ ,查询答案 answer() 的复杂度为 $O(S)$ 。则总复杂度为 $O(n \sqrt{n} (A + D) + qS)$ 。 $S$ 可以大一点,但必须保证 $A, D$ 尽可能小。

\begin{lstlisting}
struct Q { int l, r, sqrtl, id; } q[N];
int n, m, v[N], ans[N], nowans;
bool cmp(const Q &a, const Q &b) { if (a.sqrtl != b.sqrtl) return a.sqrtl < b.sqrtl; return a.r < b.r; }
void change(int x) { if (!v[x]) add(x); else del(x); v[x] ^= 1; }

for (int i=1;i<=m;i++) q[i].sqrtl = q[i].l / sqrt(n), q[i].id = i;
sort(q+1, q+m+1, cmp);
int L=1, R=0;
memset(v, 0, sizeof(v));
for (int i=1;i<=m;i++) {
	while (L<q[i].l) change(L++);
	while (L>q[i].l) change(--L);
	while (R<q[i].r) change(++R);
	while (R>q[i].r) change(R--);
	ans[q[i].id] = answer();
}
\end{lstlisting}
\subsection{树上莫队}

\begin{lstlisting}
struct Query { int l, r, id, l_group; } query[N];
int v[N], ans[N];
bool cmp(const Query &a, const Query &b) { if (a.l_group != b.l_group) return a.l_group < b.l_group; return dfn[a.r] < dfn[b.r]; }
void upd(int x) { if (!v[x]) add(x); else del(x); v[x] ^= 1; }
void go(int &u, int taru, int v) {
	int lca0 = lca(u, taru);
	int lca1 = lca(u, v); upd(lca1);
	int lca2 = lca(taru, v); upd(lca2);
	for (int x=u; x!=lca0; x=father[x]) upd(x);
	for (int x=taru; x!=lca0; x=father[x]) upd(x);
	u = taru;
}

for (int i=1;i<=m;i++) {
	if (dfn[query[i].l] > dfn[query[i].r]) swap(query[i].l, query[i].r);
	query[i].id = i; query[i].l_group = dfn[query[i].l] / sqrt(n);
}
sort(query+1, query+m+1, cmp);
int L=1,R=1; upd(1);
for (int i=1;i<=m;i++) {
	go(L,query[i].l,R);
	go(R,query[i].r,L);
	ans[query[i].id] = answer();
}
\end{lstlisting}

\section{树}

\subsection{点分治}
\begin{lstlisting}
void getsize(int x, int root = 0) {
	size[x] = 1; son[x] = 0; int dd = 0;
	for (int p = gh[x]; p; p = edge[p].next) {
		int y = edge[p].adj;
		if (y == root || vis[y]) continue;
		getsize(y, x);
		size[x] += size[y];
		if (size[y] > dd) dd = size[y], son[x] = y;
	}
}
int getroot(int x) {
	int sz = size[x];
	while (size[son[x]] > sz/2)	x = son[x]; return x;
}
void dc(int x) {
	getsize(x); x = getroot(x);
	vis[x] = 1;
	for (int p = gh[x]; p; p = edge[p].next) {
		int y = edge[p].adj;
		if (vis[y]) continue;
		dc(y);
	}
	vis[x] = 0;
}
\end{lstlisting}
\subsection{Link Cut Tree}

\ding{172} 注意,一开始必须调用 lct.init(0) ,否则求出的最小值一定会是 0 。
\ding{173} minval 维护的是\emph{链上} val 最小值。
\ding{174} sumval2 维护的是\emph{子树} val2 的和。

\begin{lstlisting}
int f[N], son[N][2], sz[N], rev[N], tot;
int val[N], minid[N], minval[N];
int val2[N], sumval2[N]; // 记得开 long long 。注意两个都要开 long long ,因为 val2 还包含了虚儿子的子树和。
stack<int> s;
void init(int i) {
	tot = max(tot, i); son[i][0] = son[i][1] = 0; f[i] = rev[i] = 0;
	if (i == 0) sz[i] = 0, val[i] = minval[i] = inf, minid[i] = i, val2[i] = sumval2[i] = 0;
	else sz[i] = 1, val[i] = minval[i] = VAL, minid[i] = i, val2[i] = sumval2[i] = VAL2;
}
bool isroot(int x) { return !f[x] || (son[f[x]][0] != x && son[f[x]][1] != x); }
void rev1(int x) { if (!x) return; swap(son[x][0], son[x][1]); rev[x] ^= 1; }
void down(int x) { if (!x) return; if (rev[x]) rev1(son[x][0]), rev1(son[x][1]), rev[x] = 0; }
void up(int x) { if (!x) return; down(son[x][0]); down(son[x][1]);
	sz[x] = sz[son[x][0]] + sz[son[x][1]] + 1; minval[x] = val[x]; minid[x] = x;
	if (minval[son[x][0]] < minval[x]) minval[x] = minval[son[x][0]], minid[x] = minid[son[x][0]];
	if (minval[son[x][1]] < minval[x]) minval[x] = minval[son[x][1]], minid[x] = minid[son[x][1]];
	sumval2[x] = sumval2[son[x][0]] + sumval2[son[x][1]] + val2[x];
}
void rotate(int x) {
	int y = f[x], w = son[y][1] == x; son[y][w] = son[x][w ^ 1];
	if (son[x][w ^ 1]) f[son[x][w ^ 1]] = y;
	if (f[y]) {
		int z = f[y];
		if (son[z][0] == y) son[z][0] = x;
		else if (son[z][1] == y) son[z][1] = x;
	}
	f[x] = f[y]; f[y] = x; son[x][w ^ 1] = y; up(y);
}
void splay(int x) {
	while (!s.empty()) s.pop(); s.push(x);
	for (int i = x; !isroot(i); i = f[i]) s.push(f[i]);
	while (!s.empty()) down(s.top()), s.pop();
	while (!isroot(x)) {
		int y = f[x];
		if (!isroot(y)) {
			if ((son[f[y]][0] == y) ^ (son[y][0] == x)) rotate(x);
			else rotate(y);
		}
		rotate(x);
	} up(x);
}
void access(int x) {for (int y = 0; x; y = x, x = f[x]) splay(x), val2[x] += sumval2[son[x][1]], son[x][1] = y, val2[x] -= sumval2[son[x][1]], up(x); }
int root(int x) { access(x); splay(x); while (son[x][0]) x = son[x][0]; return x; }
void makeroot(int x) { access(x); splay(x); rev1(x); }
void link(int x, int y) {
	makeroot(x); f[x] = y; access(x);
	// 如果需要维护子树和 val2, sumval2 ,这样是不够的。因为增加了虚边,所以需要修改 val2 值。将上面的代码替换为以下代码:
	// makeroot(x); makeroot(y); f[x] = y; val2[y] += sumval2[x]; up(y); access(x);
}
void cutf(int x) { access(x); splay(x); f[son[x][0]] = 0; son[x][0] = 0; up(x); } // 它和父亲的边
void cut(int x, int y) { makeroot(x); cutf(y); } // 切断 x 与 y 之间的边(须保证 x 与 y 相邻)
int ask(int x, int y) { makeroot(x); access(y); splay(y); return minid[y]; } // 询问 x 到 y 之间取得最小值的点
int querymin_cut(int x, int y) { int m = ask(x, y); makeroot(x); cutf(m); makeroot(y); cutf(m); return val[m]; } // 询问 x 到 y 之间取得最小值的点,并把它删去(须保证该点在 x 和 y 之间,且度数恰好为 2)
void link(int x, int y, int w) { init(++tot); val[tot] = minval[tot] = w; link(x, tot); link(y, tot); } // 在 x 和 y 之间添加一条权值为 w 的边(将边视为点插入)
int getsumval2(int x, int y) { makeroot(x); access(y); return val2[y]; } // 令 x 为根,求 y 子树的 val2 的和
\end{lstlisting}

\subsection{虚树}

设 $a[0 \cdots k - 1]$ 为需要构建虚树的点。

构建出虚树的节点保存在 $a$ 数组中, $k$ 为节点个数。加边调用函数 addedge(int x, int y, int w) 。

\begin{lstlisting}
bool cmp(int x, int y) { return dfn[x] < dfn[y]; }
stack<int> stk;
sort(a, a + k, cmp);
int m = k;
for (int j = 1; j < m; ++j)
	a[k++] = lca(a[j - 1], a[j]);
sort(a, a + k, cmp);
k = unique(a, a + k) - a;
stk.push(a[0]);
for (int j = 1; j < k; ++j) {
	int u = lca(stk.top(), a[j]);
	while (dep[stk.top()] > dep[u]) stk.pop();
	assert(stk.top() == u);
	stk.push(a[j]);
	addedge(u, a[j], dis[a[j]] - dis[u]);
}
\end{lstlisting}

\section{图}

\subsection{Tarjan 有向图强联通分量}
\ding{172} 割点的判断:一个顶点 $u$ 是割点,当且仅当满足 $(1)$ 或 $(2)$ : $(1)$ $u$ 为树根,且 $u$ 有多于一个子树(即:存在一个儿子 $v$ 使得 $dfn[u] + 1 \neq dfn[v]$ ) ; $(2)$ $u$ 不为树根,且满足存在 $(u, v)$ 为树枝边( $u$ 为 $v$ 的父亲),使得 $dfn[u] \leq low[v]$ 。
\ding{173} 桥的判断:一条无向边 $(u, v)$ 是桥,当且仅当 $(u, v)$ 为树枝边,满足 $dfn[u] < low[v]$ 。
\begin{lstlisting}
struct EDGE { int adj, next; } edge[M];
int n, m, top, gh[N];
int dfn[N], low[N], cnt, ind, stop, instack[N], stack[N], belong[N];
void addedge(int x, int y) { edge[++top].adj = y; edge[top].next = gh[x]; gh[x] = top; }
void tarjan(int x) {
	dfn[x] = low[x] = ++ind;
	instack[x] = 1; stack[++stop] = x;
	for (int p=gh[x]; p; p=edge[p].next)
		if (!dfn[edge[p].adj]) tarjan(edge[p].adj), low[x] = min(low[x], low[edge[p].adj]);
		else if (instack[edge[p].adj]) low[x] = min(low[x], dfn[edge[p].adj]);
	if (dfn[x] == low[x]) {
		++cnt; int tmp=0;
		while (tmp!=x) tmp = stack[stop--], belong[tmp] = cnt, instack[tmp] = 0;
	}
}
\end{lstlisting}

\subsection{Tarjan 双联通分量}

以下代码为点双联通分量。若要更改为边双联通,在第 8 行将 $low[next] \geq dfn[x]$ 改为 $low[next] > dfn[x]$ ,并将 14 行 $vec[tot].push\_back(x)$ 删除。

\begin{lstlisting}
void DFS(int x,int fa) {
	vis[x]=true; dfn[x]=low[x]=++times; sk[++tp]=x;
	for (int pt=first[x];pt;pt=e[pt].next) {
		int next=e[pt].to; if (e[pt].id==fa) continue;
		if (!vis[next]) {
			DFS(next,e[pt].id);
			low[x]=min(low[x],low[next]);
			if (low[next]>=dfn[x]) { // ***
				vec[++tot].clear();
				while (tp) {
					vec[tot].push_back(sk[tp--]);
					if (sk[tp+1]==next) break;
				}
				vec[tot].push_back(x); // ***
			}
		} else if (dfn[next]>last) low[x]=min(low[x],dfn[next]);
	}
}
for (i=1;i<=n;i++) if (!vis[i]) {
	DFS(i,0); last=times;
	if (tp)	{
		tot++; vec[tot].clear();
		for (i=1;i<=tp;i++) vec[tot].push_back(sk[i]);
		tp=0;
	}
}
\end{lstlisting}

\subsection{欧拉回路}
\begin{lstlisting}
struct E { int to,ne; } e[M<<1];
int t,n,m,la[N],e_top;
int in[N],out[N];
void add(int x, int y){
	out[x]++; in[y]++;
	e[++e_top]=(E){y,la[x]}; la[x]=e_top;
}
int sta[M],top;
bool vis[M<<1];
void dfs(int x){
	for(int i=la[x]; i; i=la[x]){
		la[x]=e[i].ne;
		if (vis[i]) continue;
		vis[i]=true; if (t==1) vis[i^1]=true;		
		dfs(e[i].to);
		if (t==2) sta[++top]=i;
			else sta[++top]=(i&1)?(-(i>>1)):(i>>1);
	}
}
int main(){
	scanf("%d%d%d",&t,&n,&m);
	if (m==0) YES(); if (t==1) e_top=1;
	ft(i,1,m){ scanf("%d%d",&x,&y); add(x,y); if (t==1) add(y,x); }
	if (t==1) ft(i,1,n) if (in[i]&1) NO();
	if (t==2) ft(i,1,n) if (in[i]!=out[i]) NO();
	dfs(e[3-t].to); if (top!=m) NO();
	YES(); fd(i,top,1) printf("%d ",sta[i]);
}
\end{lstlisting}

\subsection{带花树}
\begin{lstlisting}
const int N=550;
struct E { int to,ne; } e[N*N];
int n,m,la[N],e_top,f[N];
int find(int x) { return f[x]=f[x]==x?x:find(f[x]); }
int mat[N],pre[N],cond[N],q[N],l,r,vis[N],vt;
int lca(int x, int y){
	vt++; x=find(x); y=find(y);
	while (vis[x]!=vt){ if(x){vis[x]=vt;x=find(pre[mat[x]]);} swap(x,y); }
	return x;
}
void blossom(int x, int y, int g){
	while (find(x)!=g){
		pre[x]=y; if (cond[mat[x]]==1) cond[q[++r]=mat[x]]=0;
		if (f[x]==x) f[x]=g; if (f[mat[x]]==mat[x]) f[mat[x]]=g;
		y=mat[x]; x=pre[y];
	}
}
int match(int s){
	forto(i,1,n){ cond[i]=-1; pre[i]=0; f[i]=i; }
	cond[q[l=r=1]=s]=0;
	while (l<=r){ int x=q[l++];
		forE(i,x){
			int y=e[i].to;
			if (cond[y]==-1){
				if (mat[y]==0){
					while (x){
						int t=mat[x]; mat[x]=y; mat[y]=x; y=t; x=pre[y];
					}
					return true;
				}
				cond[y]=1; pre[y]=x; cond[q[++r]=mat[y]]=0;
			} else if (find(x)!=find(y) && cond[y]==0){
				int g=lca(x,y); blossom(x,y,g); blossom(y,x,g);
			}
		}
	}
	return false;
}
int main(){
	scanf("%d%d",&n,&m); int ans=0;
	while (m--){ scanf("%d%d",&x,&y); add(x,y); add(y,x); }
	forto(i,1,n) if (!mat[i] && match(i)) ans++;
	printf("%d\n",ans); forto(i,1,n) printf("%d ",mat[i]);
}
\end{lstlisting}

\subsection{KM算法}
\begin{lstlisting}
const int N=500, inf=0x7fffffff;
int n,fx[N],fy[N],pre[N];
LL w[N][N],lx[N],ly[N],sla[N];
bool vx[N],vy[N],a[N][N];
int q[N],l,r;
bool check(int x, int y){
	if (!fy[y]){
		while (x){ int t=fx[x]; fx[x]=y; fy[y]=x; y=t; x=pre[y]; }
		return true;
	}
	vy[y]=true; pre[y]=x; vx[q[++r]=fy[y]]=true; return false;
}
void bfs(int s){
	ft(i,1,n) { vx[i]=vy[i]=false; sla[i]=inf; }
	vx[q[l=r=1]=s]=true;
	while (true){
		while (l<=r){
			int x=q[l++];
			ft(y,1,n) if (!vy[y]){
				LL t=lx[x]+ly[y]-w[x][y];
				if (t==0 && check(x,y)) return;
				if (t && t<sla[y]) { sla[y]=t; pre[y]=x; }
			}
		}
		int d=inf;
		ft(y,1,n) if (!vy[y]) cmin(d,sla[y]);
		ft(x,1,n) if (vx[x]) lx[x]-=d;
		ft(y,1,n) if (vy[y]) ly[y]+=d; else sla[y]-=d;
		ft(y,1,n) if (!vy[y] && !sla[y] && check(pre[y],y)) return;
	}
}
void KM(){
	ft(x,1,n) { lx[x]=w[x][1]; ft(y,2,n) cmax(lx[x],w[x][y]); }
	ft(s,1,n) bfs(s);
}
int main(){
	int nl,nr,m; scanf("%d%d%d",&nl,&nr,&m);
	while (m--){ scanf("%d%d%d",&x,&y,&z); w[x][y]=z; a[x][y]=true; }
	n=MAX(nl,nr); KM();
	LL ans=0; ft(i,1,n) ans+=lx[i]; ft(j,1,n) ans+=ly[j];
	printf("%lld\n",ans);
	ft(i,1,nl) printf("%d ",a[i][fx[i]]?fx[i]:0);
}
\end{lstlisting}

\subsection{2-SAT}

记 $x \rightarrow y$ 的有向边表示选了 $x$ 就要选 $y$ 。

\begin{lstlisting}
struct MergePoint {
	struct EDGE { int adj, next; } edge[M];
	int ex[M], ey[M]; bool instack[N];
	int gh[N], top, dfn[N], low[N], cnt, ind, stop, stack[N], belong[N];
	void init() { cnt = ind = stop = top = 0; memset(dfn, 0, sizeof(dfn)); memset(instack, 0, sizeof(instack)); memset(gh, 0, sizeof(gh)); }
	void addedge(int x, int y) { swap(x, y); edge[++top].adj = y; edge[top].next = gh[x]; gh[x] = top; ex[top] = x; ey[top] = y; }
	void tarjan(int x) {}
	void work() { for (i) if (!dfn[i]) tarjan(i); }
} merge; 
struct Topsort {
	struct EDGE { int adj, next; } edge[M];
	int n, top, gh[N], ops[N], deg[N], ans[N]; std::queue<int> q;
	void init() { n = merge.cnt; top = 0; memset(gh, 0, sizeof(gh)); memset(deg, 0, sizeof(deg)); }
	void addedge(int x, int y) { if (x == y) return; edge[++top].adj = y; edge[top].next = gh[x]; gh[x] = top; ++deg[y]; }
	void work() {
		for (int i = 1; i <= n; ++i) if (!deg[i]) q.push(i);
		while (!q.empty()) {
			int x = q.front(); q.pop();
			for (int p = gh[x]; p; p = edge[p].next) if (!--deg[edge[p].adj]) q.push(edge[p].adj);
			if (ans[x]) continue; ans[x] = -1; ans[ops[x]] = 1; //-1 NO, 1 YES
		}
	}
} ts;
merge.init(); merge.addedge(); merge.work();
for (int i = 1; i <= n; ++i) {
	int x = merge.belong[U(i, 0)], y = merge.belong[U(i, 1)];
	if (x==y) NO(); ts.ops[x]=y; ts.ops[y]=x;
}
ts.init(); ts.work();
puts("YES"); for (int i = 1; i <= n; ++i) select(ts.ans[merge.belong[U(i,1)] == 1);
\end{lstlisting}

\subsection{网络流}
\subsubsection{最大流}
注意: \emph{top 要初始化为 1}
\begin{lstlisting}
struct EDGE { int adj, w, next; } edge[M];
int n, top, gh[N], nrl[N], dist[N], q[N];
void addedge(int x, int y, int w) { edge[++top].adj = y; edge[top].w = w; edge[top].next = gh[x]; gh[x] = top; edge[++top].adj = x; edge[top].w = 0; edge[top].next = gh[y]; gh[y] = top; }
int bfs() {
	memset(dist, 0, sizeof(dist));
	q[1] = S; int head = 0, tail = 1; dist[S] = 1;
	while (head != tail) {
		int x = q[++head];
		for (int p=gh[x]; p; p=edge[p].next)
			if (edge[p].w && !dist[edge[p].adj]) {
				dist[edge[p].adj] = dist[x] + 1;
				q[++tail] = edge[p].adj;
			}
	}
	return dist[T];
}
int dinic(int x, int delta) {
	if (x==T) return delta;
	for (int& p=nrl[x]; p && delta; p=edge[p].next)
		if (edge[p].w && dist[x]+1 == dist[edge[p].adj]) {
			int dd = dinic(edge[p].adj, min(delta, edge[p].w));
			if (!dd) continue;
			edge[p].w -= dd;
			edge[p^1].w += dd;
			return dd;
		}
	return 0;
}
int ans = 0; while (bfs()) { memcpy(nrl, gh, sizeof(gh)); int t; while (t = dinic(S, inf)) ans += t; } return ans;
\end{lstlisting}
\subsubsection{上下界有源汇网络流}

\ding{172}$T$ 向 $S$ 连容量为正无穷的边,将有源汇转化为无源汇。
\ding{173}每条边容量减去下界,设 $in[i]$ 表示流入 $i$ 的下界之和减去流出 $i$ 的下界之和。  
\ding{174}新建超级源汇 $SS, TT$ ,对于 $in[i] > 0$ 的点, $SS$ 向 $i$ 连容量为 $in[i]$ 的边。对于 $in[i] < 0$ 的点, $i$ 向 $TT$ 连容量为 $-in[i]$ 的边。 
\ding{175}求出以 $SS, TT$ 为源汇的最大流,如果等于 $\Sigma in[i] (in[i]>0)$ ,则存在可行流。再求出 $S, T$ 为源汇的最大流即为最大流。 
\ding{176}费用流:建完图后等价于求以 $SS, TT$ 为源汇的费用流。

\subsubsection{费用流}
注意: \emph{top 要初始化为 1}
\begin{lstlisting}
struct EDGE { int adj, w, cost, next; } edge[M*2];
int gh[N], q[N], dist[N], v[N], pre[N], prev[N], top, S, T;
void addedge(int x, int y, int w, int cost) {x->y(w,cost); y->x(0,-cost);}
void clear() { top = 1; memset(gh, 0, sizeof(gh)); }
bool spfa() {
	memset(dist, 63, sizeof(dist)); memset(v, 0, sizeof(v));
	int head = 0, tail = 1; q[1] = S; v[S] = 1; dist[S] = 0;
	while (head != tail) {
		(head += 1) %= N; int x = q[head]; v[x] = 0;
		for (int p=gh[x]; p; p=edge[p].next)
			if (edge[p].w && dist[x] + edge[p].cost < dist[edge[p].adj]) {
				dist[edge[p].adj] = dist[x] + edge[p].cost;
				pre[edge[p].adj] = x; prev[edge[p].adj] = p;
				if (!v[edge[p].adj]) {
					v[edge[p].adj] = 1;
					(tail += 1) %= N; q[tail] = edge[p].adj;
				}
			}
	}
	return dist[T] != inf;
}
int ans = 0;
while (spfa()) {
	int mx = inf;
	for (int x=T;x!=S;x=pre[x]) mx = min(edge[prev[x]].w, mx);
	ans += dist[T] * mx; 
	for (int x=T;x!=S;x=pre[x]) edge[prev[x]].w -= mx, edge[prev[x]^1].w += mx;
}
return ans;
\end{lstlisting}

\section{杂项}

\subsection{读入优化}

int rd(int \&x); 读入一个整数,保存在变量 x 中。如正常读入,返回值为 1 ,否则返回 EOF (-1)

\begin{lstlisting}
#define rd RD<int>
#define rdll RD<long long>
const int S = 2000000; // 2MB
char s[S], *h = s+S, *t = h;
inline char getchr(void) {
	if(h == t) { if (t != s + S) return EOF; t = s + fread(s, 1, S, stdin); h = s; }
	return *h++;
}
template <class T>
inline int RD(T &x) {
	char c = 0; int sign = 0;
	for (; !isdigit(c); c = getchr()) {
		if (c == EOF) return -1; if (c == '-') sign ^= 1;
	}
	x = 0; for (; isdigit(c); c = getchr()) x = x * 10 + c - '0';
	if (sign) x = -x; return 1;
}
\end{lstlisting}

\subsection{Vim}

\begin{lstlisting}
syntax on
set cindent
set nu
set tabstop=4
set shiftwidth=4
set background=dark

inoremap <C-j> <down>
inoremap <C-k> <up>
inoremap <C-h> <left>
inoremap <C-l> <right>
\end{lstlisting}

\subsection{Java}

\begin{lstlisting}[language=java]
头文件
import java.math.*;
import java.util.*;
public class Main {
	public static void main(String []args) {
	}
}
输入输出
Scanner cin = new Scanner(System.in);
int a = cin.nextInt();
BigDecimal a = cin.nextBigDecimal();
while (cin.hasNext()) {} // 输入到 EOF 结束
System.out.println(str); // 有换行
System.out.print(str); // 无换行
System.out.println("Hello, %s. Next year, you'll be %d", name, age); // C风格输出
大数类
BigInteger a = BigInteger.valueOf(12);
BigInteger b = new BigInteger(String.valueOf(12));
BigDecimal c = BigDecimal.valueOf(12.0);
BigDecimal d = new BigDecimal("12.0"); // 字符串防止double精度误差
大数比较
c.compareTo(BigDecimal.ZERO)==0 //判断相等,c==0
c.compareTo(BigDecimal.ZERO)>0 //判断大于,c>0
c.compareTo(BigDecimal.ZERO)<0 //判断小于,c<0
大数基本运算
Big*** add(Big*** b) // 加上b
Big*** subtract(Big*** b) // 减去b
Big*** multiply(Big*** b) // 乘b
Big*** divide(Big*** b) // 除以b
BigDecimal divide(BigDecimal b, int 精确位数, BigDecimal.ROUND_HALF_UP); // 除以b,保留小数
Big*** pow(int b) // this^b
Big*** remainder(Big*** b) // mod b
Big*** abs() // 绝对值
Big*** negate() // 取负号
Big*** max(Big*** b) // 返回this和b中的最大值
Big*** min(Big*** b) // 返回this和b中的最小值
BigInteger gcd(BigInteger val) // 返回abs(this)和abs(val)的最大公约数
BigInteger mod(BigInteger val) // 求 this mod val
BigInteger modInverse(BigInteger val) // 求逆元,返回 this^(-1) mod val
大数格式控制
toString()将BigDecimal转成字符串,然后配合一些字符串函数进行处理:
str.startWith("0"); // 以0开始
str.endWith("0"); // 以0结束
str.subString(int x, int y); // 从x到y的str的子串
str.subString(int x); // 从x到结尾的子串
c.stripTrailingZeros().toPlainString(); // c去除末尾0,转成普通字符串
setScale(int newScale, RoundingMode roundingMode) 返回BigDecimal。newScale表示保留位数。CEILING/DOWN/FLOOR/HALF_DOWN/HALF_UP。
大数进制转换
支持2~36进制(0-9 + 小写a-z)
BigInteger a=cin.nextBigInteger(2); // 读入一个二进制数
System.out.println(a.toString(2)); // 输出二进制
\end{lstlisting}

\end{document}
















