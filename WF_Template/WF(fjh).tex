\documentclass{article}
\author{范俭豪}
\title{ACM模板}
\usepackage{CJK}
\usepackage{listings}
\usepackage{geometry}
\usepackage{amssymb}
\usepackage{ulem}
\usepackage{float}
\usepackage{graphicx}
\usepackage{indentfirst}
\lstset{language=c}
\lstset{breaklines}
\lstset{extendedchars=false}
\lstset{
    numbers=left,
    numberstyle= \tiny,
    frame=shadowbox, % 阴影效果
    escapeinside=``, % 英文分号中可写入中文
    xleftmargin=2em,xrightmargin=2em, aboveskip=1em,
    framexleftmargin=2em
}
\geometry{left=3cm,right=2.5cm,top=2.5cm,bottom=2.5cm}
\begin{document}
\begin{CJK*}{UTF8}{song}
\tableofcontents
\section{数据结构}
\subsection{点分治}
\begin{lstlisting}
void Size(int x,int fa)
{
int pt,next;
    size[x]=1;
    for (pt=first[x];pt;pt=e[pt].next)
    {
        next=e[pt].to;
        if  (next==fa||vis[next]) continue;
        Size(next,x);
        size[x]+=size[next];
    }
}
int Center(int x,int fa,int ori)
{
int pt,next;
    for (pt=first[x];pt;pt=e[pt].next)
    {
        next=e[pt].to;
        if  (next==fa||vis[next]||2*size[next]<=size[ori]) continue;
        return Center(next,x,ori);
    }
    return x;
}
void DFS(int x)
{
int pt,next,i,j,low,high;
    Size(x,x);
    x=Center(x,x,x);
    vis[x]=true;
    for (pt=first[x];pt;pt=e[pt].next)
    {
        next=e[pt].to;
        if  (vis[next]) continue;
        DFS(next);
    }
    vis[x]=false;
}
\end{lstlisting}
\subsection{AC自动机}
\begin{lstlisting}
void Build_AC()
{
int  low,high,i;
Node *temp;
     low=0;
     high=-1;
     for (i=0;i<26;i++)
     if  (root->son[i]) que[++high]=root->son[i];
     for (;low<=high;low++)
         for (i=0;i<26;i++)
         if  (que[low]->son[i])
         {
             que[++high]=que[low]->son[i];
             for (temp=que[low];temp!=root;temp=temp->fail)
             if  (temp->fail->son[i])
             {
                 que[high]->fail=temp->fail->son[i];
                 break;
             }
         }
}
\end{lstlisting}
\subsection{后缀数组}
\begin{lstlisting}
bool Same(int *st,int a,int b,int len)
{
     return st[a]==st[b]&&st[a+len]==st[b+len];
}
void DA(int m=1000)
{
int  cnt,i,len,*x=arr1,*y=arr2;
     memset(arr1,127,sizeof(arr1));
     memset(arr2,127,sizeof(arr2));
     for (i=0;i<=m;i++) sum[i]=0;
     for (i=1;i<=n;i++) sum[x[i]=st[i]]++;
     for (i=1;i<=m;i++) sum[i]+=sum[i-1];
     for (i=n;i>=1;i--) sa[sum[x[i]]--]=i;
     for (cnt=0,len=1;cnt<n;len<<=1,m=cnt)
     {
         for (cnt=0,i=n-len+1;i<=n;i++) y[++cnt]=i;
         for (i=1;i<=n;i++)
         if  (sa[i]>len) y[++cnt]=sa[i]-len;
         for (i=0;i<=m;i++) sum[i]=0;
         for (i=1;i<=n;i++) sum[x[y[i]]]++;
         for (i=1;i<=m;i++) sum[i]+=sum[i-1];
         for (i=n;i>=1;i--) sa[sum[x[y[i]]]--]=y[i];
         swap(x,y);
         for (cnt=1,x[sa[1]]=1,i=2;i<=n;i++)
         if  (Same(y,sa[i-1],sa[i],len)) x[sa[i]]=cnt; else x[sa[i]]=++cnt;
     }
     for (i=1;i<=n;i++) rank[sa[i]]=i;
}
void Height()
{
int  len,i;
     for (len=0,i=1;i<=n;height[rank[i++]]=len)
     {
         if  (len) len--;
         if  (rank[i]==1) continue;
         for (;st[i+len]==st[sa[rank[i]-1]+len];len++);
     }
}
\end{lstlisting}
\section{数学}
\subsection{线性基}
\begin{lstlisting}
#include<bits/stdc++.h>
using namespace std;
#define B 30
#define N 10050

const int allset=(1<<B)-1;

int a[N];

struct LB
{
       int mat[B],cnt;
       multiset<int> st;
       LB(){}
       void clear()
       {
            st.clear();
            cnt=0;
            memset(mat,0,sizeof(mat));
       }
       void add(int x)
       {
       int  i,j;
            for (i=B-1;i>=0;i--)
            if  ((x>>i)&1)
            {
                if  (mat[i]) x^=mat[i]; else
                {
                    cnt++;
                    mat[i]=x;
                    break;
                }
            }
       }
       void fix()
       {
       int  i,j;
            for (i=0;i<B;i++)
            if  (mat[i])
            {
                for (j=i+1;j<B;j++)
                if  ((mat[j]>>i)&1) mat[j]^=mat[i];
            }
       }
       void preset()//正确性待定
       {
       int  i;
            fix();
            for (i=0;i<B;i++)
            if  (mat[i]) st.insert(mat[i]);
       }
       int  kth(int k)//正确性待定
       {
       int  i,ans;
       multiset<int>::iterator it;
            if  (k<=0||k>(1<<cnt)-1) return 0;//无解
            for (ans=i=0,it=st.begin();it!=st.end();it++,i++)
            if  ((k>>i)&1) ans^=(*it);
            return ans;
       }
       int  getmax()
       {
       int  i,ans;
            fix();
            ans=0;
            for (i=B-1;i>=0;i--)
            if  (ans^mat[i]>ans) ans^=mat[i];
            return ans;
       }
}      tree[N*10];

\end{lstlisting}
\subsection{类欧几里得}


$
\sum_{x=0}^{n} x^{k1} (\left \lfloor \frac{ax+b}{c} \right \rfloor)^{k2}
$


%\sum_{x=0}^{n} \qquad
%$x^{k1}$${frac{ax+b}{c}}^{k2}$
\begin{lstlisting}
#include<bits/stdc++.h>
using namespace std;
#define pb push_back
typedef long long ll;
const int MOD=1e9+7;
inline ll qp(ll a,ll b)
{
    ll x=1; a%=MOD;
    while(b)
    {
        if(b&1) x=x*a%MOD;
        a=a*a%MOD; b>>=1;
    }
    return x;
}
namespace Lagrange {
ll x[23333],y[23333],a[23333],g[23333],h[23333],p[23333]; int N;
void work()
{
    for(int i=0;i<N;++i) a[i]=0;
    g[0]=1;
    for(int i=0;i<N;++i)
    {
        for(int _=0;_<=i;++_)
            h[_+1]=g[_]; h[0]=0;
        for(int _=0;_<=i;++_)
            h[_]=(h[_]-g[_]*(ll)x[i])%MOD;
        for(int _=0;_<=i+1;++_) g[_]=h[_];
    }
    for(int i=0;i<N;++i)
    {
        for(int j=0;j<=N;++j) p[j]=g[j];
        for(int j=N;j;--j)
            p[j-1]=(p[j-1]+p[j]*(ll)x[i])%MOD;
        ll s=1;
        for(int j=0;j<N;++j) if(i!=j)
            s=s*(x[i]-x[j])%MOD;
        s=y[i]*qp(s,MOD-2)%MOD;
        for(int _=0;_<N;++_)
            a[_]=(a[_]+p[_+1]*s)%MOD;
    }
}
vector<int> feed(vector<int> v)
{
    N=v.size();
    for(int i=0;i<N;++i) x[i]=i,y[i]=v[i];
    work(); v.clear();
    for(int i=0;i<N;++i) v.pb(a[i]);
    while(v.size()&&!v.back()) v.pop_back();
    return v;
}
ll calc(vector<int>&v,ll xx)
{
    ll s=0,gg=1; xx%=MOD;
    for(int i=0;i<N;++i)
        s=(s+gg*v[i])%MOD,gg=gg*xx%MOD;
    return s;
}
}
using Lagrange::feed;
using Lagrange::calc;
//ps[k]=\sum_{i=0}^x i^k
vector<int> ps[2333];
//rs[k]=\sum_{i=0}^x ((i+1)^k-i^k)
vector<int> rs[2333];
struct arr{ll p[11][11];};
ll C[233][233];
arr calc(ll a,ll b,ll c,ll n)
{
    arr w;
    if(n==0) a=0;
    if(a==0||a*n+b<c)
    {
        for(int i=0;i<=10;++i)
        {
            ll t=calc(ps[i],n),s=b/c;
            for(int j=0;i+j<=10;++j)
                w.p[i][j]=t,t=t*s%MOD;
        }
        return w;
    }
    for(int i=0;i<=10;++i)
        w.p[i][0]=calc(ps[i],n);
    if(a>=c||b>=c)
    {
        arr t=calc(a%c,b%c,c,n);
        ll p=a/c,q=b/c;
        for(int i=0;i<=10;++i)
            for(int j=1;i+j<=10;++j)
            {
                ll s=0,px=1;
                for(int x=0;x<=j;++x,px=px*p%MOD)
                {
                    ll qy=1;
                    for(int y=0;x+y<=j;++y,qy=qy*q%MOD)
                    {
                        //x^(i) (px)^x q^y ??^(j-x-y)
                        s+=px*qy%MOD*C[j][x]%MOD*C[j-x][y]
                        %MOD*t.p[i+x][j-x-y]; s%=MOD;
                    }
                }
                w.p[i][j]=s;
            }
        return w;
    }
    ll m=(a*n+b)/c;
    arr t=calc(c,c-b-1,a,m-1);
    for(int i=0;i<=10;++i)
        for(int j=1;i+j<=10;++j)
        {
            ll s=calc(rs[j],m-1)*calc(ps[i],n)%MOD;
            for(int p=0;p<j;++p)
            {
                for(unsigned q=0;q<ps[i].size();++q)
                {
                    ll v=C[j][p]*ps[i][q]%MOD;
                    //v*t^p*((tc+c-b-1)/a)^q
                    s-=t.p[p][q]*v; s%=MOD;
                }
            }
            w.p[i][j]=s%MOD;
        }
    return w;
}
int T,n,a,b,c,k1,k2;
int main()
{
    freopen("a.in","r",stdin);
    freopen("a.out","w",stdout);
    for(int i=0;i<=230;++i)
    {
        C[i][0]=1;
        for(int j=1;j<=i;++j)
            C[i][j]=(C[i-1][j-1]+C[i-1][j])%MOD;
    }
    for(int i=0;i<=10;++i)
    {
        ll sp=0,sr=0; vector<int> p,r;
        for(int j=0;j<=20;++j)
            sp+=qp(j,i),sr+=qp(j+1,i)-qp(j,i),
            sp%=MOD,sr%=MOD,p.pb(sp),r.pb(sr);
        ps[i]=feed(p); rs[i]=feed(r);
    }
    scanf("%d",&T);
    while(T--)
    {
        scanf("%d%d%d%d%d%d",
        &n,&a,&b,&c,&k1,&k2);
        arr s=calc(a,b,c,n);
        int p=s.p[k1][k2];
        p=(p%MOD+MOD)%MOD;
        printf("%d\n",p);
    }
}
\end{lstlisting}
\subsection{高斯消元法实数方程}
\begin{lstlisting}
void Gauss(int n,int m)
{
int  i,j,k,t;
double mul;
     for (i=j=1;i<=n&&j<=m;i++,j++)
     {
         for (k=i+1;k<=n;k++)
         if  (abs(mat[k][j])>abs(mat[i][j]))
             for (t=1;t<=m+1;t++) swap(mat[i][t],mat[k][t]);
         if  (abs(mat[i][j])<eps)
         {
             i--;
             continue;
         }
         for (k=i+1;k<=n;k++)
         {
             mul=mat[k][j]/mat[i][j];
             for (t=1;t<=m+1;t++) mat[k][t]-=mat[i][t]*mul;
         }
     }
     for (i=n;i>=1;i--)//solved表示那个变量是否确定
     {
         for (j=1;j<=m;j++)
         if  (abs(mat[i][j])>eps) break;
         if  (j>m) continue;
         solved[j]=true;
         ans[j]=mat[i][m+1];
         for (k=j+1;k<=m;k++)
         if  (abs(mat[i][k])>eps&&!solved[k]) solved[j]=false;
         for (k=j+1;k<=m;k++) ans[j]-=ans[k]*mat[i][k];
         ans[j]/=mat[i][j];
     }
}
\end{lstlisting}
\subsection{高斯消元法模方程}
\begin{lstlisting}
long long Pow(long long x,long long y)
{
     if  (y==0) return 1;
long long t=Pow(x,y/2);
     if  (y&1) return t*t%mod*x%mod;
     return t*t%mod;
}
void Gauss(long long n,long long m)
{
long long i,j,k,t,lcm,muli,mulk;
     for (i=j=1;i<=n&&j<=m;i++,j++)
     {
         for (k=i;k<=n;k++)
         if  (mat[k][j])
         {
             for (t=1;t<=m+1;t++) swap(mat[k][t],mat[i][t]);
             break;
         }
         if  (mat[i][j]==0)
         {
             i--;
             continue;
         }
         for (k=i+1;k<=n;k++)
         if  (mat[k][j])
         {
             lcm=mat[k][j]*mat[i][j]/Gcd(mat[k][j],mat[i][j]);
             muli=lcm/mat[i][j];
             mulk=lcm/mat[k][j];
             for (t=1;t<=m+1;t++)
             {
                 mat[k][t]=mat[k][t]*mulk-mat[i][t]*muli;
                 mat[k][t]=(mat[k][t]%mod+mod)%mod;
             }
         }
     }
     for (i=n;i>=1;i--)
     {
         for (j=1;j<=m;j++)
         if  (mat[i][j]) break;
         if  (j>m) continue;
         ans[j]=mat[i][m+1];
         for (k=j+1;k<=m;k++) ans[j]-=ans[k]*mat[i][k];
         ans[j]=(ans[j]*Pow(mat[i][j],mod-2)%mod+mod)%mod;
     }
}
\end{lstlisting}
\subsection{扩展欧几里得}
求一组整数解且防爆
\begin{lstlisting}
void GCD_EX(long long A, long long &x, long long B, long long &y, long long C)
{
long long xx, yy, temp, gcd;
     if  (B == 0)
     {
         x = C;
         y = 0;
     }   else
     {
         GCD_EX(B, xx, A % B, yy, C);
         x = yy;
         temp = 1 - x / C * A;
         gcd = GCD(temp, B);
         temp /= gcd;
         B /= gcd;
         y = C / B * temp;
     }
     return;
}
\end{lstlisting}
\subsection{FWT异或,与,或}
\begin{lstlisting}
void fwtXor(int* a, int len) {
    if(len == 1) return;
    int h = len >> 1;
    fwtXor(a, h);
    fwtXor(a + h, h);
    for(int i = 0; i < h; ++i) {
        int x1 = a[i];
        int x2 = a[i + h];
        a[i] = (x1 + x2) % mod;
        a[i + h] = (x1 - x2 + mod) % mod;
    }
}
void ifwtXor(int* a, int len) {
    if(len == 1) return;
    int h = len >> 1;
    for(int i = 0; i < h; ++i) {
        int y1 = a[i];
        int y2 = a[i + h];
        a[i] = 1ll*(y1 + y2) * div2 % mod;
        a[i + h] = 1ll*(y1 - y2 + mod) * div2 % mod;
    }
    ifwtXor(a, h);
    ifwtXor(a + h, h);
}
void fwtAnd(int* a, int len) {
    if(len == 1) return;
    int h = len >> 1;
    fwtAnd(a, h);
    fwtAnd(a + h, h);
    for(int i = 0; i < h; ++i) {
        int x1 = a[i];
        int x2 = a[i + h];
        a[i] = (x1 + x2) % mod;
        a[i + h] = x2;
    }
}
void ifwtAnd(int* a, int len) {
    if(len == 1) return;
    int h = len >> 1;
    for(int i = 0; i < h; ++i) {
        int y1 = a[i];
        int y2 = a[i + h];
        a[i] = (y1 - y2 + mod) % mod;
        a[i + h] = y2;
    }
    ifwtAnd(a, h);
    ifwtAnd(a + h, h);
}
void fwtOr(int* a, int len) {
    if(len == 1) return;
    int h = len >> 1;
    fwtOr(a, h);
    fwtOr(a + h, h);
    for(int i = 0; i < h; ++i) {
        int x1 = a[i];
        int x2 = a[i + h];
        a[i] = x1;
        a[i + h] = (x1 + x2) % mod;
    }
}
void ifwtOr(int* a, int len) {
    if(len == 1) return;
    int h = len >> 1;
    for(int i = 0; i < h; ++i) {
        int y1 = a[i];
        int y2 = a[i + h];
        a[i] = y1;
        a[i + h] = (y2 - y1 + mod) % mod;
    }
    ifwtOr(a, h);
    ifwtOr(a + h, h);
}
\end{lstlisting}
\section{图论}
\subsection{tarjan}
\subsubsection{有向图强连通分量}
\begin{lstlisting}
void DFS(int x)
{
int	pt,next;
	dfn[x]=low[x]=++times;
	sk[++tp]=x;
	instack[x]=true;
	for (pt=first[x];pt;pt=e[pt].next)
	{
		next=e[pt].to;
		if  (!dfn[next])
        {
            DFS(next);
            low[x]=min(low[x],low[next]);
        }   else
        if  (instack[next]) low[x]=min(low[x],dfn[next]);
	}
	if  (low[x]==dfn[x])
	{
		tot++;
		for (;tp;)
		{
		    instack[sk[tp]]=false;
			belong[sk[tp--]]=tot;
			if  (sk[tp+1]==x) break;
		}
	}
}
for (i=1;i<=n;i++)
if  (!dfn[i]) DFS(i);
\end{lstlisting}
\subsubsection{点双联通分量}
\begin{lstlisting}
void DFS(int x,int fa)
{
int  pt,next;
     vis[x]=true;
     dfn[x]=low[x]=++times;
     sk[++tp]=x;
     for (pt=first[x];pt;pt=e[pt].next)
     {
         next=e[pt].to;
         if  (e[pt].id==fa) continue;
         if  (!vis[next])
         {
             DFS(next,e[pt].id);
             low[x]=min(low[x],low[next]);
             if  (low[next]>=dfn[x])
             {
                 tot++;
                 vec[tot].clear();
                 for (;tp;)
                 {
                     vec[tot].push_back(sk[tp]);
                     tp--;
                     if  (sk[tp+1]==next) break;
                 }
                 vec[tot].push_back(x);
             }
         }   else
         if  (dfn[next]>last) low[x]=min(low[x],dfn[next]);
     }
}
for (i=1;i<=n;i++)
if  (!vis[i])
{
   DFS(i,0);
   last=times;
   if  (tp)
   {
       tot++;
       vec[tot].clear();
       for (i=1;i<=tp;i++) vec[tot].push_back(sk[i]);
       tp=0;
   }
}
\end{lstlisting}
\subsubsection{边双联通分量}
\begin{lstlisting}
void DFS(int x,int fa)
{
int  pt,next;
     vis[x]=true;
     dfn[x]=low[x]=++times;
     sk[++tp]=x;
     for (pt=first[x];pt;pt=e[pt].next)
     {
         next=e[pt].to;
         if  (e[pt].id==fa) continue;
         if  (!vis[next])
         {
             DFS(next,e[pt].id);
             low[x]=min(low[x],low[next]);
             if  (low[next]>dfn[x])
             {
                 tot++;
                 vec[tot].clear();
                 for (;tp;)
                 {
                     vec[tot].push_back(sk[tp]);
                     tp--;
                     if  (sk[tp+1]==next) break;
                 }
             }
         }   else
         if  (dfn[next]>last) low[x]=min(low[x],dfn[next]);
     }
}
for (i=1;i<=n;i++)
if  (!vis[i])
{
   DFS(i,0);
   last=times;
   if  (tp)
   {
       tot++;
       vec[tot].clear();
       for (i=1;i<=tp;i++) vec[tot].push_back(sk[i]);
       tp=0;
   }
}
\end{lstlisting}
\subsection{网络流Dinic}
\begin{lstlisting}
struct Edge
{
       int to,flow,next;
}      e[1000050];
void Add(int x,int y,int z)
{
     e[++now].to=y;
     e[now].flow=z;
     e[now].next=first[x];
     first[x]=now;
}
bool Level()
{
int  low,high,pt,next;
     memset(level,-1,sizeof(level));
     q[low=high=0]=S;
     level[S]=0;
     for (;low<=high;low++)
         for (pt=first[q[low]];pt!=-1;pt=e[pt].next)
         {
             next=e[pt].to;
             if  (level[next]!=-1||e[pt].flow<=0) continue;
             level[next]=level[q[low]]+1;
             q[++high]=next;
             if  (next==T) return true;
         }
     return false;
}
int  Find(int x,int delta)
{
int  pt,next,temp,res=0;
     if  (x==T||delta<=0) return delta;
     for (pt=last[x];pt!=-1;last[x]=pt=e[pt].next)
     {
         next=e[pt].to;
         if  (level[next]!=level[x]+1) continue;
         temp=Find(next,min(delta,e[pt].flow));
         delta-=temp;
         res+=temp;
         e[pt].flow-=temp;
         e[pt^1].flow+=temp;
         if  (delta<=0) return res;
     }
     return res;
}
\end{lstlisting}
\subsection{费用流}
\begin{lstlisting}
struct Edge
{
       int to,flow,cost,next;
}      e[1000050];
void Add(int x,int y,int flow,int cost)
{
     e[++now].to=y;
     e[now].flow=flow;
     e[now].cost=cost;
     e[now].next=first[x];
     first[x]=now;
}
bool Find()
{
int  low,high,pt,next;
     memset(inq,false,sizeof(inq));
     memset(f,127,sizeof(f));
     q[low=high=0]=S;
     f[S]=0;
     inq[S]=true;
     for (;low<=high;inq[q[low++]]=false)
         for (pt=first[q[low]];pt!=-1;pt=e[pt].next)
         {
             next=e[pt].to;
             if  (f[next]<=f[q[low]]+e[pt].cost||e[pt].flow<=0||q[low]==T||next==S) continue;
             f[next]=f[q[low]]+e[pt].cost;
             path[next]=q[low];
             number[next]=pt;
             if  (!inq[next])
             {
                 inq[next]=true;
                 q[++high]=next;
             }
         }
     return f[T]<999999999;
}
void Addflow()
{
int  t=999999999,i;
     for (i=T;i!=S;i=path[i]) t=min(t,e[number[i]].flow);
     mincost+=t*f[T];
     maxflow+=t;
     for (i=T;i!=S;i=path[i])
     {
         e[number[i]].flow-=t;
         e[number[i]^1].flow+=t;
     }
}
\end{lstlisting}
\section{小算法}
\subsection{跨平台大随机数}
\begin{lstlisting}
int Rand()
{
    int ra = rand() % 32768;
    int rb = rand() % 32768;
    return (ra << 15) | rb;
}
\end{lstlisting}
\subsection{读入优化}
\begin{lstlisting}
char *st,*nd,ch[1000050];
inline char Get()
{
       if  (st==nd)
       {
           st=ch;
           nd=st+fread(ch,1,1000007,stdin);
       }
       return *(st++);
}
inline int Read()
{
register int    t=0;
register char c=Get();
register bool fu=false;
     for (;c!='-'&&(c<'0'||'9'<c);c=Get());
     if  (c=='-')
     {
         fu=true;
         c=Get();
     }
     for (;'0'<=c&&c<='9';c=Get()) t=10*t+c-'0';
     if  (fu) return -t;
     return t;
}
\end{lstlisting}
\subsection{KMP算法}
\begin{lstlisting}
          len = strlen(st);
          memset(prep, -1, sizeof(prep));
          memset(number, 0, sizeof(number));
          for (i = 1; i < len; i++)
          {
              for (j = i - 1; j != -1 && st[prep[j] + 1] != st[i]; j = prep[j]);
              prep[i] = j != -1 ? prep[j] + 1 : -1;
          }
\end{lstlisting}
\subsection{扩展KMP}
\begin{lstlisting}
        extend[1] = number;
        for (far = 1; far < number; far++)
        if  (tested[far] != tested[far + 1])
        {
            break;
        }
        extend[source = 2] = far - 1;
        for (j = 3; j <= number; j++)
        {
            temp = extend[j - source + 1];
            if (j + temp - 1 < far)
            {
               extend[j] = temp;
            }  else
            {
               for (k = max(far + 1, j); k <= number; k++)
               if  (tested[k] != tested[k - j + 1])
               {
                   break;
               }
               far = k - 1;
               source = j;
               extend[j] = far - source + 1;
            }
        }
\end{lstlisting}
\section{计算几何}
\subsection{凸包}
\begin{lstlisting}
bool cmp(const Point &a,const Point &b)
{
     return F(a.x-b.x)<0||F(a.x-b.x)==0&&a.y<b.y;
}
void Gram(int id[],int n)
{
int  i,mid;
     sort(id,id+n,cmp);
     tp=0;
     //凸包从x最小的点出发,逆时针方向
     for (i=0;i<n;i++)
     {
         for (;tp>=2&&Cross(p[sk[tp-1]]-p[sk[tp-2]],p[id[i]]-p[sk[tp-1]])<=0;tp--);
         //有重点必须使用小于等于不留共线点,无重点使用小于等于不留共线点,无重点使用小于留共线点
         sk[tp++]=id[i];
     }
     mid=tp;
     for (i=n-2;i>=0;i--)
     {
         for (;tp>mid&&Cross(p[sk[tp-1]]-p[sk[tp-2]],p[id[i]]-p[sk[tp-1]])<=0;tp--);
         //有重点必须使用小于等于不留共线点,无重点使用小于等于不留共线点,无重点使用小于留共线点
         sk[tp++]=id[i];
     }
     if  (n>1) tp--;
}
\end{lstlisting}
\subsection{定义}
\begin{lstlisting}
struct Point
{
       double x,y;
       Point(){}
       Point(double _x,double _y):x(_x),y(_y){}
};
struct Seg
{
    Point a,b;
    Seg(){}
    Seg(Point _a,Point _b):a(_a),b(_b){}
};
struct Circle
{
       double x,y,r;
       Point pt() {return Point(x,y);}
       double Area() {return pi*r*r;}
};
Point operator +(const Point &a,const Point &b)
{
    return Point(a.x+b.x,a.y+b.y);
}
Point operator -(const Point &a,const Point &b)
{
    return Point(a.x-b.x,a.y-b.y);
}
Point operator *(const Point &a,double b)
{
    return Point(a.x*b,a.y*b);
}
Point operator /(const Point &a,double b)
{
    return Point(a.x/b,a.y/b);
}
int F(double x)
{
    if (x>eps) return 1;
    if (x<-eps) return -1;
    return 0;
}
bool operator ==(const Point &a,const Point &b)
{
    return F(a.x-b.x)==0&&F(a.y-b.y)==0;
}
double Dist(const Point &a)
{
    return sqrt(a.x*a.x+a.y*a.y);
}
double Dot(const Point &a,const Point &b)
{
    return a.x*b.x+a.y*b.y;
}
double Cross(const Point &a,const Point &b)
{
    return a.x*b.y-a.y*b.x;
}
Point Rotate(const Point &p,double a) // 逆时针旋转
{
    return Point(p.x*cos(a)-p.y*sin(a),p.x*sin(a)+p.y*cos(a));
}
Point Inter(Seg a,Seg b)   // 两线段相交(前提有交点)
{
double s=Cross(a.b-a.a,b.a-a.a),t=Cross(a.b-a.a,b.b-a.a);
       return b.a+(b.b-b.a)*s/(s-t);
}
vector<Point> SegCir(Seg seg,Point pt,double r)//线圆
{
vector<Point> ans;
double mul;
Point vec,mid;
	ans.clear();
	vec=Rotate(seg.b-seg.a,pi/2);
	mid=Inter(seg,Seg(pt,pt+vec));
	if  (F(Dist(pt-mid)-r)>0) return ans;
	if  (F(Dist(pt-mid)-r)==0)
	{
		ans.push_back(mid);
		ans.push_back(mid);
		return ans;
	}
	vec=seg.b-seg.a;
	mul=sqrt(r*r-Dist2(mid-pt))/Dist(vec);
	ans.push_back(mid+vec*mul);
	ans.push_back(mid-vec*mul);
	return ans;
}
vector<Point> Circir(Circle a,Circle b)//圆圆相交
{
vector<Point> ans;
double dis,dis2,alpha;
Point pa,pb,vec;
      ans.clear();
      if  (a.r<b.r) swap(a,b);
      pa=a.pt();
      pb=b.pt();
      vec=pb-pa;
      dis=Dist(vec);
      dis2=Dist2(vec);
      if  (F(dis-(a.r+b.r))>0||F(dis-(a.r-b.r))<0) return ans;
      if  (F(dis-(a.r+b.r))==0)
      {
          ans.push_back(pa+vec*a.r/(a.r+b.r));
          return ans;
      }
      if  (F(dis-(a.r-b.r))==0)
      {
          ans.push_back(pa+vec*a.r/(a.r-b.r));
          return ans;
      }
      alpha=acos((a.r*a.r+dis2-b.r*b.r)/2/a.r/dis);
      ans.push_back(pa+Rotate(vec,alpha)*a.r/dis);
      ans.push_back(pa+Rotate(vec,-alpha)*a.r/dis);
      return ans;
}
double Bing(double ra,double rb,double dis)
{
double alpha,beta;
       if  (ra<rb) swap(ra,rb);
       if  (F(dis-(ra-rb))<=0) return pi*ra*ra;
       if  (F(dis-(ra+rb))>=0) return pi*ra*ra+pi*rb*rb;
       alpha=acos((ra*ra+dis*dis-rb*rb)/2/dis/ra);
       beta=acos((rb*rb+dis*dis-ra*ra)/2/dis/rb);
       return (pi-alpha)*ra*ra+(pi-beta)*rb*rb+ra*dis*sin(alpha);
}

double Jiao(double ra,double rb,double dis)
{
       return pi*ra*ra+pi*rb*rb-Bing(ra,rb,dis);
}

Point Gongmid(Circle a,Circle b)//正确性待定
{
Point pa=a.pt(),pb=b.pt();
      return pa+(pb-pa)*a.r/(a.r+b.r);
}

Point Gongright(Circle a,Circle b)
{
Point pa=a.pt(),pb=b.pt();
      return pa+(pb-pa)*a.r/(a.r-b.r);
}

int Ptinpol(Point pt)
{
int i,k,d1,d2,wn=0;
    for(i=0;i<n;i++)
    {
        if(Ins(pt,Seg(p[i],p[(i+1)%n]))) return 2;
        k=F(Cross(p[(i+1)%n]-p[i],pt-p[i]));
        d1=F(p[i].y-pt.y);
        d2=F(p[(i+1)%n].y-pt.y);
        if(k>0&&d1<=0&&d2>0)wn++;
        if(k<0&&d2<=0&&d1>0)wn--;
    }
    return wn!=0;
}
bool Cirinpol(Point pt)//需要点在多边形内的前提
{
int  i;
double nearest;
     nearest=1e+100;
     for (i=0;i<n;i++)
     {
         nearest=min(nearest,Dist(p[i]-pt));
         if  (F(Dot(pt-p[i],p[(i+1)%n]-p[i]))>0&&
             F(Dot(pt-p[(i+1)%n],p[i]-p[(i+1)%n]))>0)
             nearest=min(nearest,abs(Cross(p[i]-pt,p[(i+1)%n]-pt))/dis[i]);
     }
     return F(nearest-r)>=0;
}
bool Ins(const Point &p,const Seg &s)
{
     return F(Cross(s.a-p,s.b-p))==0&&
            F(p.x-min(s.a.x,s.b.x))>=0&&
            F(p.x-max(s.a.x,s.b.x))<=0&&
            F(p.y-min(s.a.y,s.b.y))>=0&&
            F(p.y-max(s.a.y,s.b.y))<=0;
}
double PS(const Point &p,const Seg &s)    点到线段最短距离
{
	if  (F(Dot(p-s.a,s.b-s.a))<0||F(Dot(p-s.b,s.a-s.b))<0) return min(Dist(p-s.a),Dist(p-s.b));
	return abs(Cross(s.a-p,s.b-p))/Dist(s.a-s.b);
}
double SS(const Seg &a,const Seg &b)       线段到线段最短距离
{
       return min(min(PS(a.a,b),PS(a.b,b)),min(PS(b.a,a),PS(b.b,a)));
}
double Alpha(Point a,Point b)
{
double ans;
       ans=atan2(b.y,b.x)-atan2(a.y,a.x);
       if  (ans<0) ans=-ans;
       if  (ans>pi) ans=2*pi-ans;
       return ans;
}
double Shan(Circle c,double a)
{
       return c.r*c.r*a/2;
}
\end{lstlisting}
\subsection{半平面交}
\begin{lstlisting}

bool Cmphp(Seg a,Seg b)
{
Point va,vb;
double dega,degb;
     va=a.b-a.a;
     vb=b.b-b.a;
     dega=atan2(va.y,va.x);
     degb=atan2(vb.y,vb.x);
     return F(dega-degb)<0||F(dega-degb)==0&&Cross(a.b-a.a,b.a-a.a)<0;
}
void HalfPlane(Seg hp[],int n,Point pol[],int &pols)
{
int  tp,i,low,high;
Point mid;
     hp[n++]=Seg(Point(-oo,-oo),Point(oo,-oo));
     hp[n++]=Seg(Point(oo,-oo),Point(oo,oo));
     hp[n++]=Seg(Point(oo,oo),Point(-oo,oo));
     hp[n++]=Seg(Point(-oo,oo),Point(-oo,-oo));
     sort(hp,hp+n,Cmphp);
     tp=0;//sk 0~tp-1
     low=0;
     high=-1;
     for (i=0;i<n;i++)
     if  (high-low+1==0||F(Cross(sk[high].b-sk[high].a,hp[i].b-hp[i].a)))
     {
         for (;low<high;high--)
         {
             mid=Inter(sk[high],sk[high-1]);
             if  (F(Cross(hp[i].b-hp[i].a,mid-hp[i].a))>0) break;
         }
         for (;low<high;low++)
         {
             mid=Inter(sk[low],sk[low+1]);
             if  (F(Cross(hp[i].b-hp[i].a,mid-hp[i].a))>0) break;
         }
         sk[++high]=hp[i];
     }
     for (;low<high;high--)
     {
         mid=Inter(sk[high],sk[high-1]);
         if  (Cross(sk[low].b-sk[low].a,mid-sk[low].a)>0) break;
     }
     tp=high-low+1;
     for (i=0;i<tp;i++) sk[i]=sk[low+i];
     pols=0;
     if  (tp<=2) return;
     for (i=0;i<tp;i++) pol[pols++]=Inter(sk[i],sk[(i+1)%tp]);
}

\end{lstlisting}
\subsection{圆与多边形交集}
\begin{lstlisting}
double CT(Circle c,Point a,Point b)           圆与三角形交(多边形)
{
double da,db;
Seg    s;
vector <Point> temp;
       da=Dist(a-c.pt());
       db=Dist(b-c.pt());
       if  (da>db)
       {
           swap(a,b);
           swap(da,db);
       }
       s=Seg(a,b);
       temp=CS(c,s);
       if  (F(db-c.r)<=0) return 0.5*abs(Cross(a-c.pt(),b-c.pt()));
       if  (F(da-c.r)<0)
       {
           if  (F(Dot(a-temp[1],b-temp[1]))<0) swap(temp[0],temp[1]);
           return Shan(c,Alpha(temp[0]-c.pt(),b-c.pt()))+0.5*abs(Cross(a-c.pt(),temp[0]-c.pt()));
       }
       if  (!temp.size()) return Shan(c,Alpha(a-c.pt(),b-c.pt()));
       if  (Ins(temp[1],s)&&Dist2(a-temp[1])<Dist2(a-temp[0])) swap(temp[0],temp[1]);
       if  (Ins(temp[0],s)&&Ins(temp[1],s))
       {
           return Shan(c,Alpha(a-c.pt(),temp[0]-c.pt()))+
                  Shan(c,Alpha(b-c.pt(),temp[1]-c.pt()))+
                  0.5*abs(Cross(temp[0]-c.pt(),temp[1]-c.pt()));
       }
       return Shan(c,Alpha(a-c.pt(),b-c.pt()));
}
\end{lstlisting}
\subsection{三角形面积并}
\begin{lstlisting}
#include<math.h>
#include<stdio.h>
#include<string.h>
#include<algorithm>
#define N 333
#define pr pair<ld,ld>
using namespace std;
typedef long double ld;
const ld EPS=1e-8;
const ld INF=1e100;
struct Point
{
    ld x,y;
    Point(){}
    Point(ld _,ld __):x(_),y(__){}
    void read()
    {
        double _,__;
        scanf("%lf%lf",&_,&__);
        x=_,y=__;
    }
    friend bool operator <(Point a,Point b)
    {
        if(fabs(a.x-b.x)<EPS)
        return a.y<b.y;
        return a.x<b.x;
    }
    friend Point operator +(Point a,Point b)
    {
        return Point(a.x+b.x,a.y+b.y);
    }
    friend Point operator -(Point a,Point b)
    {
        return Point(a.x-b.x,a.y-b.y);
    }
    friend Point operator *(ld a,Point b)
    {
        return Point(a*b.x,a*b.y);
    }
    friend ld operator *(Point a,Point b)
    {
        return a.x*b.x+a.y*b.y;
    }
    friend ld operator ^(Point a,Point b)
    {
        return a.x*b.y-a.y*b.x;
    }
}a[N][3],Poi[N*N];
struct Line
{
    Point p,v;
    Line(){}
    Line(Point _,Point __){p=_,v=__-_;}
    Point operator [](int k)
    {
        if(k)   return p+v;
        else    return p;
    }
    friend bool Cross(Line a,Line b)
    {
        return (a.v^b[0]-a.p)*(a.v^b[1]-a.p)<-EPS&&(b.v^a[0]-b.p)*(b.v^a[1]-b.p)<-EPS;
    }
    friend Point getP(Line a,Line b)
    {
        Point u=a.p-b.p;
        ld temp=(b.v^u)/(a.v^b.v);
        return a.p+temp*a.v;
    }
}l[N][3],T;
pr p[N];
int main()
{
    int n,m,i,j,k,x,y,cnt,tot;
    ld ans,last,A,B,sum;
    scanf("%d",&n);
    for(i=1,tot=0;i<=n;i++)
    {
        a[i][0].read(),a[i][1].read(),a[i][2].read();
        Poi[++tot]=a[i][0],Poi[++tot]=a[i][1],Poi[++tot]=a[i][2];
        sort(a[i],a[i]+3);
        if((a[i][2]-a[i][0]^a[i][1]-a[i][0])>EPS)
            l[i][0]=Line(a[i][0],a[i][2]),l[i][1]=Line(a[i][2],a[i][1]),l[i][2]=Line(a[i][1],a[i][0]);
        else
            l[i][0]=Line(a[i][2],a[i][0]),l[i][1]=Line(a[i][1],a[i][2]),l[i][2]=Line(a[i][0],a[i][1]);
    }
    for(i=1;i<=n;i++)
    {
        for(j=1;j<i;j++)
        {
            for(x=0;x<3;x++)
                for(y=0;y<3;y++)
                {
                    if(Cross(l[i][x],l[j][y]))
                        Poi[++tot]=getP(l[i][x],l[j][y]);
                }
        }
    }
    sort(Poi+1,Poi+tot+1);
    ans=0,last=Poi[1].x;
    T=Line(Point(0,-INF),Point(0,INF));
    for(i=2;i<=tot;i++)
    {
        T.p.x=(last+Poi[i].x)/2;
        for(j=1,cnt=0;j<=n;j++)
        {
            if(Cross(l[j][0],T))
            {
                A=getP(l[j][0],T).y;
                if(Cross(l[j][1],T))
                    B=getP(l[j][1],T).y;
                else
                    B=getP(l[j][2],T).y;
                if(A>B)  swap(A,B);
                p[++cnt]=pr(A,B);
            }
        }
        sort(p+1,p+cnt+1);
        for(j=1,sum=0,A=-INF;j<=cnt;j++)
        {
            if(p[j].first>A)
            {
                sum+=p[j].second-p[j].first;
                A=p[j].second;
            }
            else
            {
                if(p[j].second>A)
                    sum+=p[j].second-A,A=p[j].second;
            }
        }
        ans+=(Poi[i].x-last)*sum;
        last=Poi[i].x;
    }
    printf("%.2lf\n",(double)ans);
    return 0;
}
\end{lstlisting}
\subsection{K圆并}
\begin{lstlisting}
#include <cstdio>
#include <cstdlib>
#include <climits>
#include <iostream>
#include <algorithm>
#include <cstring>
#include <string>
#include <queue>
#include <map>
#include <vector>
#include <bitset>
#include <cmath>
#include <set>
#include <utility>
#include <ctime>
#define sqr(x) ((x)*(x))
using namespace std;

const int N = 1010;
const double eps = 1e-8;
const double pi = acos(-1.0);
double area[N];
int n;

int dcmp(double x) {
    if (x < -eps) return -1; else return x > eps;
}

struct cp {
    double x, y, r, angle;
    int d;
    cp(){}
    cp(double xx, double yy, double ang = 0, int t = 0) {
        x = xx;  y = yy;  angle = ang;  d = t;
    }
    void get() {
        scanf("%lf%lf%lf", &x, &y, &r);
        d = 1;
    }
}cir[N], tp[N * 2];

double dis(cp a, cp b) {
    return sqrt(sqr(a.x - b.x) + sqr(a.y - b.y));
}

double cross(cp p0, cp p1, cp p2) {
    return (p1.x - p0.x) * (p2.y - p0.y) - (p1.y - p0.y) * (p2.x - p0.x);
}

int CirCrossCir(cp p1, double r1, cp p2, double r2, cp &cp1, cp &cp2) {
    double mx = p2.x - p1.x, sx = p2.x + p1.x, mx2 = mx * mx;
    double my = p2.y - p1.y, sy = p2.y + p1.y, my2 = my * my;
    double sq = mx2 + my2, d = -(sq - sqr(r1 - r2)) * (sq - sqr(r1 + r2));
    if (d + eps < 0) return 0; if (d < eps) d = 0; else d = sqrt(d);
    double x = mx * ((r1 + r2) * (r1 - r2) + mx * sx) + sx * my2;
    double y = my * ((r1 + r2) * (r1 - r2) + my * sy) + sy * mx2;
    double dx = mx * d, dy = my * d; sq *= 2;
    cp1.x = (x - dy) / sq; cp1.y = (y + dx) / sq;
    cp2.x = (x + dy) / sq; cp2.y = (y - dx) / sq;
    if (d > eps) return 2; else return 1;
}

bool circmp(const cp& u, const cp& v) {
    return dcmp(u.r - v.r) < 0;
}

bool cmp(const cp& u, const cp& v) {
    if (dcmp(u.angle - v.angle)) return u.angle < v.angle;
    return u.d > v.d;
}

double calc(cp cir, cp cp1, cp cp2) {
    double ans = (cp2.angle - cp1.angle) * sqr(cir.r)
        - cross(cir, cp1, cp2) + cross(cp(0, 0), cp1, cp2);
    return ans / 2;
}

void CirUnion(cp cir[], int n) {
    cp cp1, cp2;
    sort(cir, cir + n, circmp);
    for (int i = 0; i < n; ++i)
        for (int j = i + 1; j < n; ++j)
            if (dcmp(dis(cir[i], cir[j]) + cir[i].r - cir[j].r) <= 0)
                cir[i].d++;
    for (int i = 0; i < n; ++i) {
        int tn = 0, cnt = 0;
        for (int j = 0; j < n; ++j) {
            if (i == j) continue;
            if (CirCrossCir(cir[i], cir[i].r, cir[j], cir[j].r,
                cp2, cp1) < 2) continue;
            cp1.angle = atan2(cp1.y - cir[i].y, cp1.x - cir[i].x);
            cp2.angle = atan2(cp2.y - cir[i].y, cp2.x - cir[i].x);
            cp1.d = 1;    tp[tn++] = cp1;
            cp2.d = -1;   tp[tn++] = cp2;
            if (dcmp(cp1.angle - cp2.angle) > 0) cnt++;
        }
        tp[tn++] = cp(cir[i].x - cir[i].r, cir[i].y, pi, -cnt);
        tp[tn++] = cp(cir[i].x - cir[i].r, cir[i].y, -pi, cnt);
        sort(tp, tp + tn, cmp);
        int p, s = cir[i].d + tp[0].d;
        for (int j = 1; j < tn; ++j) {
            p = s;  s += tp[j].d;
            area[p] += calc(cir[i], tp[j - 1], tp[j]);
        }
    }
}

void solve() {
    for (int i = 0; i < n; ++i)
        cir[i].get();
    memset(area, 0, sizeof(area));
    CirUnion(cir, n);
    for (int i = 1; i <= n; ++i) {
        area[i] -= area[i + 1];
        printf("[%d] = %.3lf\n", i, area[i]);
    }
}

int main() {
    freopen("a.in", "r", stdin);
    while (scanf("%d", &n) != EOF) {
        solve();
    }
    return 0;
}
\end{lstlisting}
\section{三维计算几何}
\subsection{基本定义}
\begin{lstlisting}
Point Cross(Point a,Point b)
{
      return Point(a.y*b.z-a.z*b.y,a.z*b.x-a.x*b.z,a.x*b.y-a.y*b.x);
}
double Crossxy(Point a,Point b)
{
       return a.x*b.y-a.y*b.x;
}
vector<Point> SegPlane(Seg seg,Plane p)
{
double        s,t;
Point         fa;
vector<Point> ans;
              ans.clear();
              fa=Cross(p.b-p.a,p.c-p.a);
              if  (F(Dot(fa,seg.b-seg.a))==0) return ans;
              s=Dot(p.a-seg.a,fa)/Dist(fa);
              t=Dot(p.a-seg.b,fa)/Dist(fa);
              ans.push_back(seg.a+(seg.b-seg.a)*s/(s-t));
              return ans;
}

\end{lstlisting}
\subsection{一些补充}
\begin{lstlisting}
\\ mixed product
double Mix(Point3 a,Point3 b,Point3 c)
{
	return Dot(Cross(a,b),c);
}
\\ distance from point to plane
double PP(Point3 pt,Plane pl)
{
Point3 fa=Cross(pl.b-pl.a,pl.c-pl.a);
	return abs(Dot(fa,pt-pl.a))/Dist(fa);
}
\\ get the center point from 3D(need plane well prepared)
Point3 Getcenter(Point3 p[],int n,Plane pp[],int nn)
{
int i;
double sumv,tempv;
Point3 sum;
	sumv=0;
	sum=Point3(0,0,0);
	for (i=0;i<nn;i++)
	{
		tempv=Mix(pp[i].b-pp[i].a,pp[i].c-pp[i].a,Point3(0,0,0)-pp[i].a);
		sum=sum+(pp[i].a+pp[i].b+pp[i].c)*tempv/4.0;
		sumv+=tempv;
	}
	return sum/sumv;
}
\end{lstlisting}
\section{bitset用法}
\begin{lstlisting}
C++ bitset 用法


C++的 bitset 在 bitset 头文件中,它是一种类似数组的结构,它的每一个元素只能是0或1,每个元素仅用1bit空间。

下面是具体用法

构造函数
bitset常用构造函数有四种,如下

复制代码
    bitset<4> bitset1;  //无参构造,长度为4,默认每一位为0

    bitset<8> bitset2(12);  //长度为8,二进制保存,前面用0补充

    string s = "100101";
    bitset<10> bitset3(s);  //长度为10,前面用0补充

    char s2[] = "10101";
    bitset<13> bitset4(s2);  //长度为13,前面用0补充

    cout << bitset1 << endl;  //0000
    cout << bitset2 << endl;  //00001100
    cout << bitset3 << endl;  //0000100101
    cout << bitset4 << endl;  //0000000010101
复制代码


注意:

用字符串构造时,字符串只能包含 '0' 或 '1' ,否则会抛出异常。

构造时,需在<>中表明bitset 的大小(即size)。

在进行有参构造时,若参数的二进制表示比bitset的size小,则在前面用0补充(如上面的栗子);若比bitsize大,参数为整数时取后面部分,参数为字符串时取前面部分(如下面栗子):

复制代码
    bitset<2> bitset1(12);  //12的二进制为1100(长度为4),但bitset1的size=2,只取后面部分,即00

    string s = "100101";  
    bitset<4> bitset2(s);  //s的size=6,而bitset的size=4,只取前面部分,即1001

    char s2[] = "11101";
    bitset<4> bitset3(s2);  //与bitset2同理,只取前面部分,即1110

    cout << bitset1 << endl;  //00
    cout << bitset2 << endl;  //1001
    cout << bitset3 << endl;  //1110
复制代码


可用的操作符
bitset对于二进制有位操作符,具体如下

复制代码
    bitset<4> foo (string("1001"));
    bitset<4> bar (string("0011"));

    cout << (foo^=bar) << endl;       // 1010 (foo对bar按位异或后赋值给foo)
    cout << (foo&=bar) << endl;       // 0010 (按位与后赋值给foo)
    cout << (foo|=bar) << endl;       // 0011 (按位或后赋值给foo)

    cout << (foo<<=2) << endl;        // 1100 (左移2位,低位补0,有自身赋值)
    cout << (foo>>=1) << endl;        // 0110 (右移1位,高位补0,有自身赋值)

    cout << (~bar) << endl;           // 1100 (按位取反)
    cout << (bar<<1) << endl;         // 0110 (左移,不赋值)
    cout << (bar>>1) << endl;         // 0001 (右移,不赋值)

    cout << (foo==bar) << endl;       // false (0110==0011为false)
    cout << (foo!=bar) << endl;       // true  (0110!=0011为true)

    cout << (foo&bar) << endl;        // 0010 (按位与,不赋值)
    cout << (foo|bar) << endl;        // 0111 (按位或,不赋值)
    cout << (foo^bar) << endl;        // 0101 (按位异或,不赋值)
复制代码
此外,可以通过 [ ] 访问元素(类似数组),注意最低位下标为0,如下:

    bitset<4> foo ("1011");

    cout << foo[0] << endl;  //1
    cout << foo[1] << endl;  //1
    cout << foo[2] << endl;  //0
当然,通过这种方式对某一位元素赋值也是可以的,栗子就不放了。



可用函数
bitset还支持一些有意思的函数,比如:

复制代码
    bitset<8> foo ("10011011");

    cout << foo.count() << endl;  //5  (count函数用来求bitset中1的位数,foo中共有5个1
    cout << foo.size() << endl;   //8  (size函数用来求bitset的大小,一共有8位

    cout << foo.test(0) << endl;  //true  (test函数用来查下标处的元素是0还是1,并返回false或true,此处foo[0]为1,返回true
    cout << foo.test(2) << endl;  //false  (同理,foo[2]为0,返回false

    cout << foo.any() << endl;  //true  (any函数检查bitset中是否有1
    cout << foo.none() << endl;  //false  (none函数检查bitset中是否没有1
    cout << foo.all() << endl;  //false  (all函数检查bitset中是全部为1

复制代码
补充说明一下:test函数会对下标越界作出检查,而通过 [ ] 访问元素却不会经过下标检查,所以,在两种方式通用的情况下,选择test函数更安全一些

另外,含有一些函数:

 

复制代码
    bitset<8> foo ("10011011");

    cout << foo.flip(2) << endl;  //10011111  (flip函数传参数时,用于将参数位取反,本行代码将foo下标2处"反转",即0变1,1变0
    cout << foo.flip() << endl;   //01100000  (flip函数不指定参数时,将bitset每一位全部取反

    cout << foo.set() << endl;    //11111111  (set函数不指定参数时,将bitset的每一位全部置为1
    cout << foo.set(3,0) << endl;  //11110111  (set函数指定两位参数时,将第一参数位的元素置为第二参数的值,本行对foo的操作相当于foo[3]=0
    cout << foo.set(3) << endl;    //11111111  (set函数只有一个参数时,将参数下标处置为1

    cout << foo.reset(4) << endl;  //11101111  (reset函数传一个参数时将参数下标处置为0
    cout << foo.reset() << endl;   //00000000  (reset函数不传参数时将bitset的每一位全部置为0
复制代码
同样,它们也都会检查下标是否越界,如果越界就会抛出异常

最后,还有一些类型转换的函数,如下:

复制代码
    bitset<8> foo ("10011011");

    string s = foo.to_string();  //将bitset转换成string类型
    unsigned long a = foo.to_ulong();  //将bitset转换成unsigned long类型
    unsigned long long b = foo.to_ullong();  //将bitset转换成unsigned long long类型

    cout << s << endl;  //10011011
    cout << a << endl;  //155
    cout << b << endl;  //155
复制代码
\end{lstlisting}


\end{CJK*}
\end{document} 